%%% A template to produce a nice-looking Curriculum Vitae.
%%% Kieran Healy <kjhealy@gmail.com>
%%% Most recent version is at http://kjhealy.github.com/kjh-vita
%%%
%%% ------------------------------------------------------------------------
%%% Requirements (should be included in a modern tex distribution):
%%% ------------------------------------------------------------------------
%%% xelatex
%%% fontspec.sty
%%% hyperrref.sty
%%% xunicode.sty
%%% color.sty
%%% url.sty
%%% fancyhdr.sty
%%%
%%% ------------------------------------------------------------------------
%%% Optional
%%% ------------------------------------------------------------------------
%%% git
%%% vc.sty
%%% revnum.sty
%%% Fonts
%%%
%%% ------------------------------------------------------------------------
%%% Note
%%%------------------------------------------------------------------------
%%% Because this is a hand-tweaked file, be on the look out for \medskip, 
%%% \bigskip and \newpage commands here and there, which are used to balance
%%% the layout or avoid widows & orphans, etc. You should of course add or 
%%% remove these as needed.
%%%------------------------------------------------------------------------

\documentclass[11pt,a4paper]{article}

%%%------------------------------------------------------------------------
%%% Metadata
%%%------------------------------------------------------------------------

%% Change as needed. Or just add me as a coauthor. Only some of these are 
%% used below in the hyperref declaration and address banner section.
\def\myauthor{James Keirstead}
\def\mytitle{Vita}
\def\mycopyright{\myauthor}
\def\mykeywords{}
\def\mybibliostyle{plain}
\def\mybibliocommand{}
\def\mysubtitle{}
\def\myaffiliation{Imperial College London}
\def\myaddress{Department of Civil and Environmental Engineering}
\def\myemail{j.keirstead@imperial.ac.uk}
\def\myweb{http://www.jameskeirstead.ca}
\def\myphone{+44 (0) 207 594 6010}
\def\myfax{+44 (0) 207 594 5934}
\def\myversion{}
\def\myrevision{}

\date{} % not used (revision control instead)
\def\mykeywords{James, Keirstead, James Keirstead, Vita, CV, Resume, Engineering}

%%%------------------------------------------------------------------------  
%%% Git version tracking 
%%%------------------------------------------------------------------------

%% If you don't use git or the vc package (from CTAN), comment this out.
%% If you comment it out, be sure to remove the \rfoot comment below, too.
%% See vc manual to compile with xelatex -enable-write18 vita
\immediate\write18{vc.bat}
\input{vc}

%%%------------------------------------------------------------------------
%%% Required style files
%%%------------------------------------------------------------------------
\usepackage{url,fancyhdr}
%%\usepackage{revnum} % for reverse-numbered publications (revnumerate environment) if needed.

%% needed for xelatex to work
\usepackage{fontspec}
\usepackage{xunicode}

%% color for the links 
\usepackage[usenames,dvipsnames]{xcolor}
\definecolor{ImperialBlue}{rgb}{0.082,0.416,0.608}
\definecolor{ImperialLightBlue}{rgb}{0.545,0.682,0.8}

\usepackage[right=0.75in, bottom=1in]{geometry}
%% hyperlinks
\usepackage[xetex, 
	colorlinks=true,
	urlcolor=ImperialLightBlue,
	plainpages=false,
  	pdfpagelabels,
  	bookmarksnumbered,
  	pdftitle={\mytitle},
  	pagebackref,
  	pdfauthor={\myauthor},
  	pdfkeywords={\mykeywords}
  	]{hyperref}

%%%------------------------------------------------------------------------
%%% Document
%%%------------------------------------------------------------------------
\begin{document}

%% Choose fonts for use with xelatex
%% Minion and Myriad are widely available, from Adobe. 
%% Pragmata is available to buy at http://www.fsd.it/fonts/pragma.htm
%% and is worth every penny. Any good monospace font will work fine, though.
%% Consolas or inconsolata are good alternatives.
\setromanfont[Mapping={tex-text},Numbers={OldStyle},Ligatures={Common}]{Minion Pro} 
\setsansfont[Mapping=tex-text,Colour=156a9b]{Myriad Pro}
\setmonofont[Mapping=tex-text,Scale=0.9]{Consolas} % Pragmata, Lucida Console


%%%------------------------------------------------------------------------
%%% Local commands
%%%------------------------------------------------------------------------

%% Marginal header
%% Note: as the document goes on you may need to introduce a (gradually increasing)
%% \vspace element to keep the marginal header pleasingly aligned with the first 
%% item in the body text. Like this: \marginhead{{\vskip 0.4em}Grants}, or 
%% \marginhead{{\vskip 0.8em}Service}. Experiment as needed.
\newcommand{\marginhead}[1]{\marginpar{\textsf{{\footnotesize #1}}}}


%% custom ampersand (font consistent with the one chosen above)
\newcommand{\amper}{{\fontspec[Scale=.95,Colour=AA0000]{Minion Pro Medium}\selectfont\&\,}}

%% No bullets on labels
\renewcommand{\labelitemi}{~} 

%% Custom hanging indent for vita items
\def\ind{\hangindent=1 true cm\hangafter=1 \noindent}
%\def\ind{\hangindent=18pt\hangafter=1 \noindent}
\def\labelitemi{~}
\renewcommand{\labelitemii}{~}

%%%------------------------------------------------------------------------
%%% Page layout
%%%------------------------------------------------------------------------
\pagestyle{fancy}
\renewcommand{\headrulewidth}{0pt}
\fancyhead{}
\fancyfoot{}
\lfoot{{\scriptsize\thepage}}

%% git revision control footer 
\rfoot{\textcolor{Gray}{\texttt{\scriptsize \VCRevision\ on \VCDateTEX}}} % git revision info inserted via external script -- see docs for vc package for details. comment out this line if you're not using vc, and also remove the \input{vc} line above.

%%%------------------------------------------------------------------------
%%% Address and contact block
%%%------------------------------------------------------------------------
\begin{minipage}[t]{2.95in}
\flushright {\footnotesize \href{http://www.imperial.ac.uk/civilengineering}{Department of Civil \& Env.\ Engineering} \\ Imperial College London \\ London, UK\\ \vspace{-0.04in} SW7 2AZ}   
\end{minipage}
\hfill     
%\begin{minipage}[t]{0.0in}
% dummy (needed here)
%\end{minipage}
\hfill
\begin{minipage}[t]{1.7in}
  \flushright \footnotesize Phone: \myphone \\ 
  Fax: \myfax  \\ 
  {\scriptsize  \texttt{\href{mailto:\myemail}{\myemail}}} \\
  {\scriptsize  \texttt{\href{\myweb}{\myweb}}}
\end{minipage}


\bigskip

\bigskip

%% Name 
\noindent{\Large {D\textsc{r} J\textsc{ames} K\textsc{eirstead}}}
\reversemarginpar
\raggedright

\bigskip

\bigskip

%% Appointments

\noindent\marginhead{Appointments}%
%
\emph{Imperial College London \vspace{0.01in}}

\ind 2011--Present. Lecturer.      

\ind 2009--2011. Research Fellow and Team Leader, BP Urban Energy Systems project.

\ind 2006--2009. Research Associate, BP Urban Energy Systems project. \vspace{0.02in}

\medskip
\emph{Independent consultancy\vspace{0.01in}}

\ind Environmental Change Institute, University of Oxford (for Carbon Trust, Department for International Development), UK Government Foresight, International Institute for Applied Systems Analysis, EDF/Weber Shandwick, The Economist Group.

\bigskip

%% Education

\noindent\marginhead{Education}%
%
\emph{University of Oxford \vspace{0.01in}}

\ind 2006.  DPhil, Energy Policy. \emph{Behavioural responses to photovoltaic systems in the UK domestic sector.}

\ind 2003.  MSc, Environmental Change and Management (Distinction).

\medskip

\noindent\emph{Queen's University, Kingston, Ontario, Canada\vspace{0.02in}}

\ind 2001. BSc Applied Science, Civil Engineering. (First Class Honours.) 

\bigskip
 
%% Publications
\noindent\marginhead{Selected\\Publications}%
%
%\medskip
\noindent\emph{Books \vspace{0.01in}}

\ind James Keirstead and Nilay Shah (editors). Forthcoming. \emph{Urban Energy Systems: an Integrated Approach}. London: Routledge.
%\normalsize

\bigskip

\emph{Journal articles \vspace{0.01in}} % 0.05
 
%% Use revnumerate environment if numbered publications are needed. 
%% (Include it above in the preamble).
%% \renewcommand{\labelenumi}{\textsc{a}\theenumi.}
%% \begin{revnumerate}

\ind James Keirstead and Aruna Sivakumar. Accepted. ``\href{http://dx.doi.org/10.1111/j.1530-9290.2012.00486.x}{Using activity-based modeling to simulate urban resource demands at high spatial and temporal resolutions}''. \emph{Journal of Industrial Ecology}.

\ind James Keirstead and Carlos Calderon.  2012. ``\href{http://dx.doi.org/10.1016/j.enpol.2012.03.058}{Capturing spatial effects, technology interactions, and uncertainty in urban energy and carbon models: retrofitting Newcastle as a case-study}.''  \emph{Energy Policy}, 46: 253--267.

\ind James Keirstead, Mark Jennings, and Aruna Sivakumar. 2012. ``\href{http://dx.doi.org/10.1016/j.rser.2012.02.047}{A review of urban energy system models: approaches, challenges, and opportunities}.'' \emph{Renewable and Sustainable Energy Reviews}, 16(6): 3847--3866.

\ind James Keirstead, Nouri Samsatli, Nilay Shah and C\'{e}line Weber. 2012. ``\href{http://dx.doi.org/10.1016/j.energy.2011.06.011}{The impact of CHP (combined heat and power) planning restrictions on the efficiency of urban energy systems}.'' \emph{Energy}, 41(1): 93--103. 

\ind James Keirstead, Nouri Samsatli, A.\ Marco Pantaleo, and Nilay Shah. 2012. ``\href{http://dx.doi.org/10.1016/j.biombioe.2012.01.022}{Evaluating biomass energy strategies for a UK eco-town with an MILP optimization model}.'' \emph{Biomass and Bioenergy}, 39: 306--316.

\ind James Keirstead and Nilay Shah. 2011. ``\href{http://dx.doi.org/10.1016/j.compenvurbsys.2010.12.005}{Calculating minimum energy urban layouts with mathematical programming and Monte Carlo analysis techniques}.'' \emph{Computers, Environment and Urban Systems} 35(5):~368--377. 

\ind James Keirstead. 2008. ``\href{http://dx.doi.org/10.1016/j.enpol.2008.09.019}{What changes, if any, would increased levels of low-carbon decentralised energy have on the built environment?}'' \emph{Energy Policy} 36(12):~4518--4521.

\ind James Keirstead. 2007. ``\href{http://dx.doi.org/10.1016/j.enpol.2007.02.019}{Behavioural responses to photovoltaic systems in the UK domestic sector}'' \emph{Energy Policy} 35(8):~4128--4141.

\bigskip

\noindent\emph{Book chapters \vspace{0.01in}}
% \renewcommand{\labelenumi}{\textsc{c}\theenumi.}
% \begin{revnumerate}

%\ind Various chapters in UES book?

\ind James Keirstead and Nilay Shah. Forthcoming (2012). ``Urban Energy Systems Planning, Design and Implementation'' in \emph{Energizing Sustainable Cities}, edited by A.\ Grubler and D.\ Fisk. London: Earthscan/Routledge.

\ind James Keirstead and Nilay Shah. Forthcoming. ``The changing role of optimization in urban planning'' in \emph{Optimization, Simulation and Control}, edited by A.\ Chinchuluun, P.M.\ Pardalos, R.\ Enkhbat, S.\ Pistikopoulos.  Springer Series in Optimization and Its Applications.  Springer.

\ind Stephen Hammer, James Keirstead, Shobhakar Dhakal, \emph{et al}. 2011. ``Climate Change and Urban Energy Systems'' in \emph{Climate Change and Cities: First Assessment Report of the Urban Climate Change Research Network}, edited by C.\ Rosenzweig, W.D.\ Solecki, S.A.\ Hammer, S.\ Mehrotra.  Cambridge: Cambridge University Press, 85--112.

%\end{revnumerate}

\bigskip 

%\noindent\emph{Conference papers \vspace{0.01in}}


\bigskip
 
%\newpage
%\noindent\emph{Reviews and other occasional writing \vspace{0.05in}}

%\renewcommand{\labelenumi}{\textsc{r}\theenumi.}
%\begin{revnumerate}

% %\end{revnumerate}
% \bigskip

%% Presentations
\noindent\marginhead{Recent\\Invited Talks}%
%
%\ind 2012. ``Modelling urban energy systems with SynCity'' BRI Issue Launch, London. September.
%
\ind 2012. ``Modelling urban energy systems with SynCity'' NSF Workshop on Sustainable Cities, Denver, USA.  August.

\ind 2012. ``Modelling approaches to urban energy systems: optimization, simulation, and more'' Martin Centre seminar series, University of Cambridge, UK. February.

\ind 2011. ``Modelling urban energy systems'' National Academy of Engineering/EU Frontiers of Engineering symposium, Irvine, CA. November.

\ind 2011. ``Approaches, challenges and opportunities in urban energy systems modelling'' IIASA, Laxenburg, Austria. March.

\ind 2010. ``Challenges in multi-scale urban energy modelling and data collection: What's `just right' for urban energy data?'' \textsc{iis} seminar, University of Tokyo. November.

%\end{revnumerate}

\bigskip

\noindent\marginhead{Grants and \newline Awards}%
%\medskip
%
\ind 2012. Energy Futures Lab Research Challenge, Imperial College.  Principal investigator (£8,000).

\ind 2011. Biomass system value-chain modelling, Energy Technologies Institute.  Co-investigator (£153,000).

\ind 2011. Energy systems analysis of South Heaton, Newcastle University.  Co-investigator (£2,500).

\ind 2009. MBA Innovation, Entrepreneurship and Design, Imperial College.  Ideator (30 person-months research time).

\bigskip 

\noindent\marginhead{Service to the \newline Profession}%
%
\ind 2011--present. Board member, Sustainable Urban Systems section, International Society for Industrial Ecology.

\ind 2011--present. Technical advisory panel member, Challenging Lock-in Through Urban Energy Systems EPRSC project.

\ind 2011--2012. Technical committee member, CIBSE ASHRAE Technical Symposium 2012.

\ind Reviewer, Engineering and Physical Sciences Research Council, Economic and Social Research Council, and more than 15 peer-reviewed energy journals.


\bigskip

\noindent\marginhead{Teaching}%
%
% \ind 2012-present.  CI {\addfontfeatures{Numbers={Lining}}101} Drawing.
\ind 2011-present.  CI {\addfontfeatures{Numbers={Lining}}181} Energy and Infrastructure I.

\ind 2011--present.  CI{\addfontfeatures{Numbers={Lining}}9} Sustainable Development MSc module.

\ind 2007--present. Supervised or co-supervised 22 MSc projects, 5 MEng projects and internships.

\bigskip

%\noindent\marginhead{PhD students}%
%%
%\ind 2012--present.  Stefan Pfenninger. \emph{Title}

%\ind 2012--present.  Rembrandt Koppelaar. \emph{Title}

%\ind 2012--present.  Tom Ravalde? \emph{Title}

\noindent\marginhead{Affiliations}%
%
\ind 2010--present.  Member and Chartered Energy Engineer, Energy Institute.

\ind 2010--present.  Member, International Society for Industrial Ecology.

\ind 2008--present.  Member, British and International Institutes for Energy Economics.

\end{document}
