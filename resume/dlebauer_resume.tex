\documentclass[a4paper,10pt]{article}

\usepackage[margin=0.75in]{geometry}

%A Few Useful Packages
\usepackage{marvosym}
\usepackage{fontspec} 					%for loading fonts
\usepackage{xunicode,xltxtra,url,parskip} 	%other packages for formatting
\RequirePackage{color,graphicx}
\usepackage[usenames,dvipsnames]{xcolor}
\usepackage[big]{layaureo} 				%better formatting of the A4 page
% an alternative to Layaureo can be ** \usepackage{fullpage} **
\usepackage{supertabular} 				%for Grades
\usepackage{titlesec}					%custom \section

%Setup hyperref package, and colours for links
\usepackage{hyperref}
\definecolor{linkcolour}{rgb}{0,0.2,0.6}
\hypersetup{colorlinks,breaklinks,urlcolor=linkcolour, linkcolor=linkcolour}

%FONTS
\defaultfontfeatures{Mapping=tex-text}
%\setmainfont[SmallCapsFont = Fontin SmallCaps]{Fontin}
%%% modified for Karol Kozioł for ShareLaTeX use
\setmainfont[
SmallCapsFont = Fontin-SmallCaps.otf,
BoldFont = Fontin-Bold.otf,
ItalicFont = Fontin-Italic.otf
]
{Fontin.otf}
%%%

%CV Sections inspired by: 
%http://stefano.italians.nl/archives/26
\titleformat{\section}{\Large\scshape\raggedright}{}{0em}{}[\titlerule]
\titlespacing{\section}{0pt}{3pt}{3pt}
%Tweak a bit the top margin
\addtolength{\voffset}{-1.3cm}


%-------------WATERMARK TEST [**not part of a CV**]---------------
\usepackage[absolute]{textpos}

\setlength{\TPHorizModule}{30mm}
\setlength{\TPVertModule}{\TPHorizModule}
\textblockorigin{2mm}{0.65\paperheight}
\setlength{\parindent}{0pt}

%--------------------BEGIN DOCUMENT----------------------
\begin{document}

%WATERMARK TEST [**not part of a CV**]---------------
%\font\wm=''Baskerville:color=787878'' at 8pt
%\font\wmweb=''Baskerville:color=FF1493'' at 8pt
%{\wm 
%	\begin{textblock}{1}(0,0)
%		\rotatebox{-90}{\parbox{500mm}{
%			Typeset by Alessandro Plasmati with \XeTeX\  \today\ for 
%			{\wmweb \href{http://www.aleplasmati.comuv.com}{aleplasmati.comuv.com}}
%		}
%	}
%	\end{textblock}
%}

\pagestyle{empty} % non-numbered pages

\font\fb=''[cmr10]'' %for use with \LaTeX command

%--------------------TITLE-------------
\par{\centering
		{\Huge David LeBauer, Ph.D.
	}\bigskip\par}

%--------------------SECTIONS-----------------------------------
%Section: Personal Information
\section{Personal Data}

\begin{tabular}{rl}
%    \textsc{Place and Date of Birth:} & Someplace, Italy  | dd Month 1912 \\
    \textsc{Address:}   & 2911 E Canyon Bend St, Tucson, AZ \\
    \textsc{Phone:}     & 919-275-0360\\
    \textsc{email:}     &   
        \href{mailto:dlebauer@gmail.com}{dlebauer@gmail.com}\\
    \textsc{github:}    &     
        \href{https://github.com/dlebauer}{github.com/dlebauer}\\
    \textsc{cv:}        &
        \href{https://github.com/dlebauer/vita/raw/master/dlebauer-full-vita.pdf}{dlebauer-full-vita.pdf}

\end{tabular}

%Scientist with over ten years of experience developing methods to make the best use of available information and technology in pursuit of more productive and sustainable agriculture and more efficient science. I use software and code as a shared infrastructure to formally integrate knowledge and observations across domains to create new scientific insights.
I am a scientist with over twenty years of research experience, and over ten as a project lead and manager. 
I develop methods for more productive, sustainable agriculture through more efficient research and use software, data, and collaboration to integrate knowledge across domains.

%Section: Work Experience at the top
\section{Selected Work Experience}


\begin{tabular}{lp{11cm}}

2019$\rightarrow$ & \textsc{Director of Data Science}, \textbf{Division of Agriculture Life Sciences, Cooperative Extension, University of Arizona}\\
& \footnotesize{Started a data science program for the division with a mission of enabling data intensive research. Supervised a team of 4-6 research software engineers and data scientists; collaborated on multiple multi-instituional and interdisciplinary research teams; wrote proposals, supervised many small research projects, provided on demand training and support.}\\
\multicolumn{2}{c}{} \\

2015$\rightarrow$ & \textsc{Senior Scientific Advisor}, \textbf{Agrible, Inc}\\
& \footnotesize{Designed, prototyped, and consulted on analyses of crop data and simulation models used to predict yields, to inform sustainable farm and supply chain management. }\\
\multicolumn{2}{c}{} \\


2015--2018 & \textsc{Senior Scientific Advisor}, \textbf{Agrible, Inc}\\
& \footnotesize{Designed, prototyped, and consulted on analyses of crop data and simulation models used to predict yields, to inform sustainable farm and supply chain management. }\\
\multicolumn{2}{c}{} \\

2012--2018 & \textsc{Research Scientist and Technical Manager},  \textbf{University of Illinois} \\
& \small \emph{TERRA Reference - Data and Computing Pipeline (2015$\rightarrow$)}\\
&\footnotesize{Was awarded \$1.7m from the DOE %plus computing and six petabytes of storage from NCSA 
to design and implement a pipeline for advance sensor based plant observation data stream. Supervised three full time and ten part time programmers. Coordinated development of software and data products across six collaborating research groups representing diverse domains across genomics, physiology, breeding, remote sensing, computer vision, and robotics. Coordinated feedback form end users, an advisory committee, and industry partners.
}\\
 & \emph{EBI Ecosystem Modeling Program (2012-2015)}\\
& \footnotesize{Developed, led research program, and trained scientists with a \$600k annual budget that supported three postdocs, one research programmer, a project manager, and dozens of part-time student employees. Led development and applications of simulation, statistical, and informatics software (PEcAn, BETYdb, and BioCro)}\\
\multicolumn{2}{c}{} \\
2009-2012 & \textsc{Postdoctoral Researcher}, \textbf{University of Illinois}\\
& \footnotesize{Created, implemented (as PEcAn and BETYdb), and distributed a new software platform for synthetic research.}\\\
\textsc{} & \\
\multicolumn{2}{c}{} \\


\end{tabular}


%Section: Education
\section{Education}
\begin{tabular}{rp{11cm}}	
2008 & \textsc{Ph.D. in Earth System Science}, \textbf{University of California at Irvine}\\
& \small Thesis: "Carbon Balance Under Nitrogen Enrichment", advisor: Dr. Kathleen Treseder\\
%&\footnotesize{Quantified nitrogen limitation of plant growth globally.}\\
&\small Demonstrated nitrogen limits plant growth almost everywhere; linked fungal diversity to decomposition; explored nitrogen impacts on savannah ecology and carbon cycling\\
 2001 & \textsc{M.S. Ecology, Agriculture}, \textbf{University of California at Davis} \\
 &\small Quantified microbial response to nitrogen, tomato response to mycorrhizae.\\
1998 & \textsc{B.S. Biology with honors, Ecology concentration}, \textbf{Duke University} \\
\end{tabular}

%Section: Software
\section{Software Projects}

\begin{tabular}{rp{11cm}}
TERRA REF:&TERRA Reference Phenotyping Platform\\
& \small \textit{Creator \& lead}\\
& \small Computing pipeline, reference data products, and cloud environments advancing the use of technology to improve crops and agriculture. Data and metadata products that adopt existing conventions and implement new ones that promote interoperability.\\
& \small web: \href{http://www.terraref.org}{terraref.org}; code:  \href{https://github.com/terraref}{github.com/terraref}\\
\multicolumn{2}{c}{} \\
PEcAn:&The Predictive Ecosystem Analyzer\\
& \small \textit{Creator, developer, \& lead}\\
& \small Ecoinformatics toolbox for model-data synthesis, analysis, and prediction. Integrates observations with scientific understanding of crop and ecosystem functioning as an iterative, directed learning process. Increased efficiency of science over 100x by overcoming technical obstacles. Used by many crop and ecosystem modeling groups.\\
& \small web:\href{http://www.pecanproject.org}{pecanproject.org};  code:\href{https://github.com/PecanProject/pecan}{github.com/PecanProject/pecan}\\
\multicolumn{2}{c}{} \\
BETYdb:&Biofuel Ecophysiological Traits and Yields database\\
& \small \textit{Creator \& development lead}\\
& \small{Database network and web interface to harmonize heterogeneous plant, ecosystem, and agronomic data. Supports meta-analysis, simulation modeling, and synchronization across research teams; currently used for data management by six research teams.}\\
& \small web: \href{http://www.betydb.org}{betydb.org}; code:  \href{https://github.com/PecanProject/bety}{github.com/PecanProject/bety}\\
\multicolumn{2}{c}{} \\
BioCro:&Bioenergy Crop simulator\\
& \small \textit{Development lead}\\
& \small Dynamical simulation model combining physics, chemistry and physiological processes to predict crop growth and water use. Development lead\\
&\small code: \href{https://github.com/ebimodeling/biocro}{github.com/ebimodeling/biocro}\\
\multicolumn{2}{c}{} \\
GHGVC:& Ecosystem Climate Regulation Services Calculator.\\
& \small \textit{Development lead}\\
& \footnotesize Web interface and R package to compute biophysicaland biogeochemical impacts of land use change.\\
& \small web: \href{https://www.ecosystemservicescalc.org}{www.ecosystemservicescalc.org}; code: \href{https://github.com/ebimodeling/ghgvc}{github.com/ebimodeling/ghgvc}%; R package: \href{https://github.com/ebimodeling/ghgvcR}{github.com/ebimodeling/ghgvcR}
\\\

\end{tabular}

%Section: Key Publications
%\section{Key Publications}
%\begin{tabular}{rl}
%2017 & BETYdb: a yield, trait, and ecosystem service database applied to %second-generation bioenergy feedstock production\\
%2013 & Facilitating feedbacks between field measurements and ecosystem %models, Ecological Monographs \\
%2008 & Nitrogen Limitation of Net Primary Productivity is Globally Distributed, Ecology \\


%\end{tabular}


%Section: Honors and Awards 
%\section{Honors and Awards}
%\begin{tabular}{rl}
%2014 & Faculty Fellow, National Center for Supercomputing Applications %\\
%2007 & Graduate Student Representative, Department of Earth System %Science\\
%2005 & Graduate Fellow, Kearney Soil Science Foundation\\

%\end{tabular}

\section{Skills}
\begin{tabular}{rp{11cm}}
General:& Agile project management, grant writing, communication, certified Software and Data Carpentry Instructor \\
Lab and Field:& Soil science, plant physiology, gas flux, ecology, biochemistry, microbiology\\
Data: & Visualization, harmonization, interoperability, relational, geospatial, and raster databases \\
Analysis: & Experimental design, crop and ecosystem simulation, regression, hierarchical Bayesian methods, reproducible research \\ 
Software:& R, SQL, bash\\

\end{tabular}


\end{document}
