\documentclass{xetexCV}

\cvname{David Shaner LeBauer}
\institution{University of Illinois at Urbana-Champaign}
\contactaddress{Energy Biosciences Institute\\
  1206 W. Gregory Drive\\
  Urbana, Illinois \texttt{61801}, USA}
\phonenumber{(217)\thinspace 300-0266}
\faxnumber{(217)\thinspace 244-3637}
\email{dlebauer@illinois.edu}

\hyphenpenalty=10000

\setmainfont[Ligatures={Common}, Numbers={OldStyle}]{Fontin}
\setsansfont{Fontin Sans}
\usepackage{hyperref}


\begin{document}
\makecvtitle 

% \cvsection{Summary}

%  \cvsection{Research Objectives}

% \begin{itemize}
% \item I seek to understand and predict carbon, water, and energy balance of agricultural ecosystems in the face of global change. I use and develop a combination of informatics, simulation, and statistical tools to support this research.
% \end{itemize}
%  \cvsection{Specific Objectives re: Climate Coorporation}


%     \begin{itemize}
%     \item Discuss potential for collaboration
%     \item Learn about the Quantitative Agronomist position
%     \item Invite developers to ``Advancing Software for Ecological Foreacsting'' workshop (March 2014, Urbana, IL)
%     \end{itemize}


  

% \cvsubsection{Areas of Specialization}

% biogeochemistry, ecoinformatics, software carpentry, simulation modeling

\cvsection{Employment}

%\cvsubsection{Current}

\years{2012-present} 
Research Scientist, University of Illinois\\
Manager, Energy Biosciences Institute Ecosystem Modeling Program\\
Consultant, Global Change Solutions LLC.\\



%\cvsubsection{Past}


Postdoctoral Researcher, University of Illinois.\years{2009-2012}\\ 
%Developed Scientific Workflow for Model-Data Synthesis\\
%Advisor: Mike Dietze \\


Graduate Researcher, Teaching Assistant, University of California at Irvine. \years{2004-2008}\\
%Advisor: Kathlene Treseder\\

Laboratory Manager, Agricultural Extension Educator, \years{2003-2004} North Carolina A\&T University \\

Graduate Researcher, Teaching Assistant, University of California at Davis. \years{2001-2003}\\


Laboratory and Field Technician, Duke University. \years{1996-2001}\\

\cvsection{Education}

\years{2004-2008} PhD in Earth System Science, University of California at Irvine\\  \\ % Advisor: Kathleen Treseder \\
\years{2001-2003} MS in Agricultural Ecology, University of California at Davis 
\\ \\% Advisors: Louise Jackson and Kate Scow \\
\years{1994-1998} BS in Biology, with Honors,  Duke University \\ \\% Advisor: Janis Antonovics\\


\cvsection{Relevant Skills}

Biogeochemistry and Ecosystem Ecology (Field, Lab, Theory, Computation)\\
Advanced R Programming\\
Scientific Communication\\
Crop and Ecosystem Modeling\\ Data Curation, Model-Data Synthesis, Statistics, \\ Scientific Computing, Database Design, Reproducible Research\\Software Carpentry, Agile/Scrum
% \begin{description}
% \item{\textbf{Mechanistic Models}} BioCro, Ecosystem Demography Model
% \item{\textbf{Data Collection}} Laboratory, Field, and Computational Methods in Biogeochemistry, Ecology, and Ecosystem Science; extraction and standardization of published results
% \item{\textbf{Statistics}} Experimental Design, Exploratory Analysis, Modeling, Meta-analysis, Data Assimilatoin, Visualization 
% \item{\textbf{Software Development}} bug tracking, continuous integration, literate programming, reproducible research, Scrum, unit testing, software carpentry
% \item{\textbf{Software Tools}} ArcGIS, BUGS, \LaTeX, Linux, Redmine, R, SQL
% \end{description}

\newpage
\cvsection{Software Development}

\cvsubsection{Lead Developer}
\hspace{1em}

\years{PEcAn}  The Predictive Ecosystem Analyzer\\ ecoinformatics toolbox that enables model-data synthesis and ecological prediction. \\
web:\href{http://www.pecanproject.org}{pecanproject.org} \\ 
code, wiki:\href{https://github.com/PecanProject/pecan}{github.com/PecanProject/pecan}\\


\years{BETYdb} The Biofuel Ecophysiological Traits, Yields (and other ecosystem services) database\\ open access portal to facilitate collection, synthesis, and distribution of plant trait and ecosystem service data; ecoinformatics infrastructure for PEcAn workflow \\ 
web:\href{http://www.betydb.org}{betydb.org} \\ 
code, wiki:\href{https://github.com/PecanProject/bety}{github.com/PecanProject/bety}\\

\cvsubsection{Contributor} 

\hspace{1em}

\years{BioCro \& CropCent} The BioCro crop productivity model scales from leaf physiology to ecosystem level processes to simulate productivity and water use of potential cellulosic biofuel feedstocks.\\
 CropCent is a coupled productivity-biogeochemistry model that couples BioCro to the DayCent soil biogeochemistry model\\
BioCro code: \href{https://github.com/ebimodeling/biocro}{github.com/ebimodeling/biocro}\\


\years{GHGVC} The Greenhouse Gas Value Calculator computes the value (in units of W/m2 forcing) of the net balance of greenhouse gases based on disturbance loss and annual flux following \href{http://dx.doi.org/10.1111/j.1365-2486.2010.02220.x}{ Anderson-Teixeira and DeLucia 2010}.\\
web: \href{http://pecandev.igb.illinois.edu:6480/}{pre-beta development version}\\
R package:\href{https://github.com/ebimodeling/ghgvcR}{github.com/ebimodeling/ghgvcR}\\
Ruby code: \href{https://github.com/ebimodeling/ghgvc}{github.com/ebimodeling/ghgvc}\\


\cvsection{Funding}

\textit{Advancing Software for Ecological Forecasting (Workshop)} \years{2014}\\   Chair, with Perry de Valpine (UCB) and Matthew Smith (Microsoft Research)\\ Microsoft Research Connections, DOE RCN forecast, Institute for Genomic Biology:\$30,000 \\ 

%``ADD TITLE HERE'' \years{2014} \\ co-organizer with Kiona Ogle (ASU, Chair) and Jennifer Powers (UMN) \\ \$10,000 from DOE RCN Forecast \\ 

\textit{Model-data synthesis and forecasting across the upper Midwest:  \years{2011-2014}\\ Partitioning uncertainty and environmental heterogeneity in ecosystem carbon}''  \\ co-author, with Michael Dietze, Ankur Desai, and Petr Bajscy \\  NSF Advances in Bioinformatics Infrastructure: \$770,653\\ 

\textit{Feedstock and Ecosystem Service Modeling Program}  \years{2013-2014} \\ lead-author, with Stephen Long, Evan DeLucia, and Carl Bernacchi\\
Energy Biosciences Institute: \$1,230,000\\

\textit{Grassland Carbon Dynamics along Nitrogen and Depth Gradients}  \years{2005-2007}\\
Kearney Soil Science Foundation Graduate Student Fellowship: \$34,000 \\ 

\textit{Decomposition responses to nitrogen in a California grassland} \years{2006}\\
NRS Mildred E. Mathias Graduate Student Research Grant Program: \$1,000  \\

\textit{Graduate Fellowship} \years{2004-2005}\\
UCI Department of Earth System Science: \$32,000\\

\textit{Establishment of a landscaped mushroom garden}  \years{1998} \\ Duke University Benenson Award in the Arts: \$1,500


\cvsection{Activities}


\cvsubsection{Leadership}


 Biofuel Agricultural Model Intercomparison Project Coordinator \years{2013-} \\

University of Illinois Representative \years{2012-}\\ National Ecological Observatory Network (NEON).\\

Scientist, Climate Science Experts Referral Service \years{2009-2011}, American Geophysical Union \\

\mypub  Graduate Student Representative \years{2006-2007}% \\Department of Earth System Science, University of California at Irvine 

\mypub Graduate Student Seminar Organizer \years{2005-2006}%\\Department of Earth System Science, University of California at Irvine 

\cvsubsection{Membership} American Association for the Advancement of Science\\ American Geophysical Union \\ Ecological Society of America\\ Sigma Xi


\cvsubsection{Professional Training (non-degree)}

Presenting Data and Information (Edward Tufte) \years{2013} \\

Ecosystem Demography model workshop, Harvard University \years{2012} 
\\

R Development Master Classes (Hadley Wickham)\\
\\

Flux Measurement and Modeling, Niwot Ridge, CO \years{2010}  
\\

Plant Functional Traits, NCEAS Distributed Graduate Seminar \years{2008}
  \\

Radiocarbon in Ecology and Earth System Science, UC Irvine \years{2004}
\\

Sustainable Agriculture, Central Carolina Community College \years{1999}


\cvsubsection{Peer Review}  
\begin{center}
\begin{tabular}{lll}
 Applied Soil Ecology     &  Ecology                &  Ecology Letters               \\
 Ecological Applications  &  Global Change Biology  &  Geophysical Research Letters  \\
 New Phytologist          &  Plant Soil             &  NSF                           \\
\end{tabular}
\end{center}

 \newpage
\cvsubsection{Teaching}

\years{2013-2014} Ecology in the Kitchen Workshop Series (Sauerkraut, Shiitake, and Maple Syrup production), Common Ground Food Cooperative\\

Facilitating Feedbacks between Ecosystem Models and Data (Workshop), Ecological Society of America Meetings\\

\years{2009} Global Change Biology, California Summer School for Science and Math \\

\years{1996} HsC~79.17, Project WILD Outdoor Experiential Education Course, Duke University\\

\cvsubsection{Teaching Assistant}

\href{https://sites.google.com/site/bio100lwlab14/}{Experimental Biology Laboratory}, UCI \years{Fall~2008}
\\

Atmospheric and Environmental Sciences, \years{Summer~2008} California Summer School for Science and Math\\

GIS for Environmental Science, UCI, \years{Winter~2006, 2008}\\

ESS~13: Global Change Biology, UCI, \years{Spring 2007}\\

PLP~140: Mushroom Cultivation, UCD, \years{Winter 2003}\\

Archaeology and Geology Southwest field trip, Duke Talent Identification Program, \years{Summer 1997}\\


\end{document}


%% Log of Reviews (incomplete) 
%% APSOIL
% APSOIL-D-09-00091R1
% APSOIL-D-09-00379
% APSOIL-D-11-00166
% APSOIL-D-12-00268R1
%% https://mail.google.com/mail/ca/u/0/#search/Thank+you+for+the+review+of+APSOIL
%% Ecology 
% 07-1104
% 08-0792
% 09-0935
% 10-1558
% 11-1398
% https://mail.google.com/mail/ca/u/0/#search/%22Review+Received+by+Ecology%22 
%% GRL 
% #2008GL033983
%% New Phytologist
% NPH-MS-2012-14400


\cvsection{Publications}

\cvsubsection{Journal Articles}


\years{in review / revision}  
\mypub Davis, S., \textbf{LeBauer, D.S.}, and Long, S.L. Light to liquid fuel: theoretical and realized energy conversion (invited). Journal of Experimental Botany\\

\mypub Wang, D., D. Jaiswal, \textbf{D.S. LeBauer}, T. Wertin, A. Leakey, S. Long A physiological and biophysical model of coppice willow (Salix spp.) production and predicted yields for the contiguous USA in current and future climate scenarios.\\

\mypub Dietze, M.C., Serbin, S.P., \textbf{LeBauer, D.S.}, et. al. quantitative assessment of a terrestrial biosphere model's data needs across North American biomes (invited). Journal of Geophysical Research - Biogeosciences\\

\mypub Davidson, C., Hu, F.S., \textbf{LeBauer, D.S.}, Serbin, S.P., and Dietze, M.C. Iterative Calibration of the Ecosystem Demography Model for Alaskan Tundra: Model-Data Feedbacks and Data Assimilation. \\  

\years{in press}
 \mypub Wang, D., \textbf{LeBauer, D.S.}, Kling, G. Voigt, T.,  M.C. Dietze. Ecophysiological Screening of Tree Species for Biomass Production. Ecosphere\\

\years{2013}
\mypub \textbf{LeBauer, D.S}, M.C.~Dietze, and B. Bolker. Translating Probability Distributions: From R to BUGS and Back Again\\

\mypub Dietze, M.C., \textbf{D.S.~LeBauer}, and R. Kooper. On improving the communication between models and data. Plant Cell \& Environment (invited). \doi{10.1111/pce.12043}\\

%\mypub R. Kooper, K. McHenry, M.C. Dietze, \textbf{D.S. LeBauer}, S. Serbin, A. Desai. 2013. et al Ecological Cyberinfrastructure and HPC Towards More Accurately Predicting Future Levels of Greenhouse Gases. 

\mypub Wang, D., \textbf{D.S.~LeBauer}, and M.C.~Dietze. Predicted yields of short-rotation hybrid poplar (Populus spp.) for the contiguous US. Ecological Applications. \doi{10.1890/12-0854.1}\\

\mypub \textbf{LeBauer, D.~S.}, D.~Wang,, C.~Davidson, K.~Richter, and M.~C. Dietze. Facilitating feedbacks between field measurements and ecosystem models. Ecological Monographs. \doi{10.1890/12-0854.1}\\

\years{2010} 
\mypub \textbf{LeBauer, D.~S.} Litter degradation rate and beta-glucosidase activity increase with fungal diversity. \textit{Can.\ J.\ For.\ Res.}, 40(6): 1076–-1085 \doi{10.1139/X10-054}\\

\mypub Wang, D., \textbf{D.~S. LeBauer} and M.~C. Dietze.  A quantitative review comparing the yield of switchgrass in monocultures and mixtures in relation to climate and management factors. \textit{Glob.Change Biol. Bioenergy}, 2(1):16--25 \doi{ 10.1111/j.1757-1707.2010.01035.x}\\
 
\years{2008}
\mypub Allison, S.~D, \textbf{D.~S. LeBauer}, M.~R. Ofrecio, R. Reyes, A.~M. Ta, and T.~M. Tran. Low levels of nitrogen addition stimulate decomposition by boreal forest fungi. \textit{Soil Bio. Biochem.}, 41:293--302 \\
 
\mypub \textbf{LeBauer, D.~S.} and K.~K. Treseder. Nitrogen Limitation of Net Primary Productivity in Terrestrial Ecosystems is Globally Distributed. \textit{Ecology}, 89(2): 371--379 \doi{10.1890/06-2057.1}

\years{2004}

\mypub Okano, Y., K.R.~Hirstova, C.M.~Leutenegger, L.E.~Jackson, R.F.~Denison, B.~Gebreyesus, \textbf{D.S.~LeBauer}, and K.M.~Scow. Effects of ammonium on the population size of ammonia-oxidizing bacteria in soil- Application of real-time PCR. \textit{Appl. Environ. Microb.}, 70:1008--1016 \doi{10.1128/AEM.70.2.1008-1016.2004}


% \cvsubsection{Book Chapter}

% \mypub Isikhuemhen, O.~S. and \textbf{D.~S. LeBauer} \years{2004}. Growing Pleurotus tuberregium. p270-281 \textit{in} Mushroom Grower's Handbook 1: Oyster Mushroom Cultivation. MushWorld. Seoul, Korea.

% %\cvsubsection{Invited Presentations}

% %\years{2013} 
% %TODO ORNL\\
% %TODO ESA\\

% %\years{2011} 
% %ESA \\

% \cvsubsection{Popular Press}


% \mypub \textbf{LeBauer, D.~S.} \years{2008}. Nitrogen Pollution Boosts Plant Growth. \href{http://scitizen.com/biodiversity/nitrogen-pollution-boost-plant-growth_a-22-1526.html}{Scitizen.com}
% \\

% \mypub \textbf{LeBauer, D.~S.} \years{2004}. Consider Shiitake Cultivation! Stewardship News, Carolina Farm Stewardship Association. Pittsboro, NC. 24(2):8-11.\\
