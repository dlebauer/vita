% LaTeX resume using res.cls
\documentclass[line,10pt]{res} 
%\usepackage{helvetica} % uses helvetica postscript font (download helvetica.sty)
\usepackage{newcent}   % uses new century schoolbook postscript font 
\usepackage{simplemargins}
\usepackage{hanging}

\settopmargin{1in}
\setleftmargin{0.8in}
\setrightmargin{1.5in}
\setbottommargin{0.3in}
\begin{document}

\name{David S. LeBauer as of August 31, 2009}
% \address used twice to have two lines of addresm

\begin{resume}

%Institute for Genomic Biology\hfill dlebauer@illinois.edu\\
%University of Illinois at Urbana-Champaign\hfill (949) 433-7410\\
%Urbana, IL, 61801 \\

%\section{OBJECTIVE}       A position in the field of applied ecosystem science. 
%\section{OBJECTIVE} My objective is to develop and promote sustainable land management practices that will improve the efficiency of agricultural, forest, and urban ecosystems. My background is in the function of ecosystems and their response to management and perturbation. I can help land managers develop the ecosystem services that humans require while protecting natural resources and reducing dependence on external inputs. In this way, I seek work opportunities to evaluate and develop healthy, productive ecosystems based on scientific research and predictive models.
\section{RESEARCH INTERESTS}
My research investigates the interactions between the nitrogen and carbon cycles, and how these interactions are affected by other components of global change. I am currently developing a dynamic global vegitation model that will aid in understanding the biogeochemical consequences of different cropping systems in agriculture and forestry.

\section{EDUCATION}

 \textbf{Ph.D.} Earth System Science\hfill University of California at Irvine, 2008 \\
 \textbf{M.S.} Ecology with agriculture emphasis \hfill University of California at Davis, 2003   \\ 
 \textbf{B.S.} Biology with ecology concentration, honors \hfill Duke University, 1998 

\section{PROFESSIONAL EMPLOYMENT}
 {\bf  Postdoctoral Scholar} \hfill University of Illinois, 2009 - 2011 \\Development, parameterization, analysis, and validation of ecological models relevant to the assessment and forecasting of biofuel feedstocks and their impacts on ecosystem services.\\
  Advisor: Dr. Michael Dietze

{\bf Teaching Faculty} CA State Summer School for Math and Science  \hfill UC Irvine, 2009      

{\bf  Research Assistant} \hfill UC Irvine, 2005-2008 \\
    Synthesized global patterns of plant community response to nitrogen addition. Quantified the long term, dose-dependent effects of nitrogen on carbon pools in a California grassland. Measured enzyme production and litter decomposition rates by fungal communities. \\Advisor: Dr. Kathleen Treseder

               {\bf  Teaching Assistant} \hfill \\
CA State Summer School for Math and Science  \hfill             UC Irvine, Summer 2008\\
BIO 100LW: Experimental Biology Laboratory \hfill               UC Irvine, Fall 2008\\
ESS 134: GIS for Environmental Science \hfill                   UC Irvine, Winter 2006, 2008\\
ESS 13: Global Change Biology  \hfill                           UC Irvine, Spring 2007

{\bf Laboratory and Operations Manager} \hfill NC A\&T SU, 2003-2004 \\
Cultivated and distributed mushroom innoculum to over 200 farmers statewide, led farmer workshops, promoted agricultural, industrial, and medicinal applications of white-rot fungi; advised student lab members; collected and cultured marketable wild fungi. \\Supervisor: Dr. Omon Isikhuemhen

{\bf  Research Assistant} \hfill UC Davis, 2001-2003 \\   Evaluated the production, yield, and chemistry of tomatoes with and without mycorrhizal fungi at different levels of nitrogen fertilization. Measured the relationships between nitrogen availability and nitrifier bacteria population size and nitrification rates. \\Advisors: Drs. Louise Jackson and  Kate Scow

{\bf  Teaching Assistant} PLP 140: Mushroom Cultivation \hfill UC Davis, Spring, 2003 
      
{\bf Mycological Consulting}  \hfill
                 Dr. Annie King and Dr. Jamal Assi, UC Davis 2003\\ 
\phantom{} \hfill Wildflower Organics, Dawsonville, GA, 1999-2000 

{\bf Laboratory and Field Technician} \hfill  Duke University, 1997-2001\\ Identified, mapped, measured, and cultivated forest trees; analyzed forest canopy images; managed data; collected and prepared plant and soil samples for analysis. \\Supervisors: Drs. Jacqueline Mohan, William Schlesinger, and Jim Clark
\newpage
\opening
\section{PEER-REVIEWED PUBLICATIONS}
\begin{verse}
 Allison, SD, DS LeBauer, MR Ofrecio, R Reyes, AM Ta, and TM Tran, 2008. Low levels of nitrogen addition stimulate decomposition by boreal forest fungi. Soil Biology and Biochemistry, 41:293-302 \\ 
 LeBauer, DS and KK Treseder, 2008. Nitrogen Limitation of Net Primary Productivity in Terrestrial Ecosystems is Globally Distributed. Ecology 89:371-379\\
 Okano, Y, KR Hirstova, CM Leutenegger, LE Jackson, RF Denison, B Gebreyesus, DS LeBauer, and KM Scow, 2004 Effects of ammonium on the population size of ammonia-oxidizing bacteria in soil- Application of real-time PCR. Applied and Environmental Microbiology 70:1008-1016\\
\end{verse}

\section{NON-PEER-REVIEWED PUBLICATIONS}
{\sl Book Chapter}
\begin{verse}
Isikhuemhen, OS, and DS LeBauer. 2004. Growing \textit{Pleurotus tuberregium}. p270-281 in Mushroom Grower's Handbook 1: Oyster Mushroom Cultivation. MushWorld. Seoul, Korea\\
\end{verse}
{\sl News Articles}
\begin{verse} 
 LeBauer, DS, 2008. Nitrogen Pollution Boosts Plant Growth. Scitizen.com, France\\
 LeBauer, DS, 2004. Consider Shiitake Cultivation! Stewardship News, Carolina Farm Stewardship Association. Pittsboro, NC. 24(2):8-11 \end{verse}

\section{SELECTED PRESENTATIONS AT SCIENTIFIC MEETINGS}
\begin{verse}LeBauer, D.S., and K.K. Treseder 2007. Nitrogen deposition reduces nitrogen limitation of terrestrial net primary production, ESA, San Jose, CA\\
LeBauer, D.S., and K.K. Treseder 2006. Grassland carbon dynamics along nitrogen and depth gradients. ESA, Memphis, TN\\
LeBauer, D.S. and O.S. Isikhuemhen. 2003. Small farm mushroom production for food and medicine. (Workshop) CFSA Sustainable Agriculture Conference, Rock Hill, SC\\
LeBauer, D.S., Y. Okano, K.M. Scow, and L.E. Jackson 2002. Correlating gross nitrification rates with population dynamics of ammonia oxidizers in soil. ESA, Tucson, AZ\end{verse}
\section{SERVICE, OUTREACH} 
{\bf Graduate Student Representative,} Dept. of Earth System Science, UC Irvine, 2006-2007\\  {\bf Graduate Student Seminar Organizer,} Dept. of Earth System Science, UC Irvine, 2005-2006\\ 
{\bf Undergraduate Mentor,} Dept. of Earth System Science, UC Irvine, 2005-2008\\ 
{\bf  Ad hoc reviewer,} Geophysical Research Letters, Ecology, Plant and Soil\\ 
{\bf Judge,} California State Science Fair, 2006-2009\\
{\bf Mentor,} FARMS Leadership, Inc., Sacramento, CA, 2003.

\section{PROFESSIONAL SOCIETY MEMBERSHIP} Ecological Society of America, Mycological Society of America, American Geophysical Union, Sigma Xi

\end{resume}
\end{document}
