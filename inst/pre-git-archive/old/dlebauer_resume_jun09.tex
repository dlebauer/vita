% LaTeX resume using res.cls
\documentclass[line,10pt]{res} 
%\usepackage{helvetica} % uses helvetica postscript font (download helvetica.sty)
\usepackage{newcent}   % uses new century schoolbook postscript font 
\usepackage{simplemargins}
\usepackage{hanging}

\settopmargin{1in}
\setleftmargin{0.8in}
\setrightmargin{1.5in}
\setbottommargin{0in}
\begin{document}

\name{David S. LeBauer}
% \address used twice to have two lines of addresm

 
\begin{resume}

785 Harbor Cliff Way Unit 174 \hfill dlebauer@gmail.com\\
Oceanside, CA, 92054 \hfill (760) 722-1206\\

%\section{OBJECTIVE}       A position in the field of applied ecosystem science. 
%\section{OBJECTIVE} My objective is to develop and promote sustainable land management practices that will improve the efficiency of agricultural, forest, and urban ecosystems. My background is in the function of ecosystems and their response to management and perturbation. I can help land managers develop the ecosystem services that humans require while protecting natural resources and reducing dependence on external inputs. In this way, I seek work opportunities to evaluate and develop healthy, productive ecosystems based on scientific research and predictive models.

\section{EDUCATION}

 Ph.D. Earth System Science\hfill University of California at Irvine, 2008 \\
 M.S. Ecology with agriculture emphasis \hfill University of California at Davis, 2003   \\ 
 B.S. Biology with ecology concentration, honors \hfill Duke University, 1998 

\section{EXPERIENCE}
               {\sl  Postdoctoral Scholar, } \hfill University of Illinois, 2009 - 2011 \\
  Development, parameterization, analysis, and validation of ecological models relevant to the assessment and forecasting of biofuel feedstocks and their impacts on ecosystem services.
  Advisor: Dr. Michael Dietze
		
               {\sl  Research Assistant, } \hfill UC Irvine, 2005-2008 \\
 Thesis: ``Ecosystem carbon balance under nitrogen enrichment''\\
                Synthesized global patterns of plant community response to nitrogen addition. Quantified the long term, dose-dependent effects of nitrogen on carbon pools in a California grassland. Measured enzyme production and litter decomposition rates by fungal communities. Advisor: Dr. Kathleen Treseder
              
                {\sl Laboratory and Operations Manager} \hfill NC A\&T SU, 2003-2004 \\
                 Cultivated and distributed mushroom innoculum to over 200 farmers statewide, led farmer workshops, promoted agricultural, industrial, and medicinal applications of white-rot fungi; advised student lab members; collected and cultured marketable wild fungi. Supervisor: Dr. Omon Isikhuemhen

               {\sl  Research Assistant} \hfill UC Davis, 2001-2003 \\
   Evaluated the production, yield, and chemistry of tomatoes with and without mycorrhizal fungi at different levels of nitrogen fertilization. Measured the relationships between nitrogen availability and nitrifier bacteria population size and nitrification rates. Advisors: Drs. Louise Jackson and  Kate Scow
                 
                {\sl Mycological Consulting}  \\
                 \textit{Pleurotus} cultivation on tomato pomace \hfill Dr. Annie King and Dr. Jamal Assi, UC Davis 2003\\ Shiitake cultivation \hfill Crabtree Valley Farms, Chatanooga, TN 2000\\ \phantom{} \hfill Wildflower Organics, Dawsonville, GA, 1999-2000 \\ \phantom{} \hfill Sustenance Farm, Bear Creek, NC, 1999

                 {\sl Laboratory and Field Technician} \hfill  Duke, 1997-2001\\
 Identified, mapped, measured, and cultivated forest trees; analyzed forest canopy images; managed data; collected and prepared plant and soil samples for analysis. Supervisors: Drs. Jacqueline Mohan, William Schlesinger, and Jim Clark

                 {\sl Undergraduate Research} \hfill  Duke, 1997-1998\\
        Honors Thesis: ``Mycorrhizally mediated competition between two host plants'' Advisor: Dr. Janis Antonovics.Established a landscaped mushroom garden. Advisor: Dr. Rytas Vilgalys.

\newpage
\opening

                \section{TEACHING}

BIO 100LW: Experimental Biology Laboratory \hfill               UC Irvine, Fall 2008
\\
CA State Summer School for Math and Science  \hfill             UC Irvine, Summer 2008,2009\\
ESS 134: GIS for Environmental Science \hfill                   UC Irvine, Winter 2006, 2008\\
ESS 13: Global Change Biology  \hfill                           UC Irvine, Spring 2007\\
Log Cultivation of Shiitake Mushrooms \hfill                 NC A\&T SU Agricultural Extension, 2003-2004\\
PLP 140: Mushroom Cultivation \hfill                             UC Davis, Spring 2003\\
HsC 79.17: Project WILD Outdoor Education Course \hfill        Duke University, Spring 1996\\

\section{FUNDING}

{Graduate Student Fellowship} \hfill Kearney Soil Science Foundation, 2005-2007\\
{Mildred E. Mathias Graduate Research Grant} \hfill UC Natural Reserve System, 2006\\
{Graduate Student Fellowship} \hfill Earth System Science, UC Irvine, 2004-2005\\
{Benenson Award in the Arts} \hfill Duke University,  1998\\

\section{PUBLICATIONS}
{\sl Journal Articles}
\begin{verse}
 Allison, SD, DS LeBauer, MR Ofrecio, R Reyes, AM Ta, and TM Tran, 2008. Low levels of nitrogen addition stimulate decomposition by boreal forest fungi. Soil Biology and Biochemistry, 41:293-302 \\ 
 LeBauer, DS and KK Treseder, 2008. Nitrogen Limitation of Net Primary Productivity in Terrestrial Ecosystems is Globally Distributed. Ecology 89:371-379\\
 Okano, Y, KR Hirstova, CM Leutenegger, LE Jackson, RF Denison, B Gebreyesus, DS LeBauer, and KM Scow, 2004 Effects of ammonium on the population size of ammonia-oxidizing bacteria in soil- Application of real-time PCR. Applied and Environmental Microbiology 70:1008-1016\\
\end{verse}

{\sl Book Chapter}
\begin{verse}
 Isikhuemhen, OS, and DS LeBauer. 2004. Growing \textit{Pleurotus tuberregium}. p270-281 in Mushroom Grower's Handbook 1: Oyster Mushroom Cultivation. MushWorld. Seoul, Korea\\
\end{verse}

{\sl News Articles}
\begin{verse} 
 LeBauer, DS, 2008. Nitrogen Pollution Boosts Plant Growth. Scitizen.com, France\\
 LeBauer, DS, 2004. Consider Shiitake Cultivation! Stewardship News, Carolina Farm Stewardship Association. Pittsboro, NC. 24(2):8-11 \end{verse}

\section{SERVICE} {\sl Graduate Student Representative,} 2006-2007;  {\sl Graduate Student Seminar Organizer,} 2005-2006, Dept. of Earth System Science, UC Irvine;  {\sl  Ad hoc reviewer,} Geophysical Research Letters, Ecology, Plant and Soil; {\sl Judge,} California State Science Fair, 2006-present;{\sl Mentor,} FARMS Leadership, Inc., Sacramento, CA, 2003.{\sl  Teacher,} K-12, 1998-99. Scientists in the Classroom, Sigma Xi, Durham, NC., 1998-1999

\section{PROFESSIONAL AFFILIATIONS} Ecological Society of America, Mycological Society of America, American Geophysical Union, Sigma Xi

\end{resume}
\end{document}
