 % cv-us.tex 
% $Id: cv-us.tex,v 1.28 2006/12/12 22:53:52 jrblevin Exp $
%
% LaTeX Curriculum Vitae Template
%
% Copyright (C) 2004-2006 Jason Blevins
%
% You may use use this document as a template to create your own CV
% and you may redistribute the source code freely. No attribution is
% required in any resulting documents, however, I do ask that you
% please leave this notice and the above URL in the source code if you
% choose to redistribute this file.
%
% Jason R. Blevins «jrblevin@sdf.lonestar.org»
% http://jrblevin.freeshell.org
% Durham, December 12, 2006
%
%%---------------------------------------------------------------------------%
%
% Notes:
%
% * Don't forget to change `pdfauthor' and `keywords' in the \hypersetup
% section below.
%
% * To create a new page use: \newpage \opening
%
% * res.cls includes an \address{} command which can be used up to twice,
% but my address is too long for the format it uses.
%
% * Alternate documentclass statement to put headings in margin:
% \documentclass[margin,line,11pt,final]{res}
%
% * Can divide publication/presentation list into subsections by year:
% \section{\bf\small\hspace{8mm}2006}
%
%%----------------------------------------------------------------------------%
\documentclass[overlapped,line,letterpaper,10pt]{res}
\usepackage{ifpdf}
\ifpdf
\usepackage[pdftex]{hyperref}
\else
\usepackage[hypertex]{hyperref}
\fi
\hypersetup{
letterpaper,
colorlinks,
urlcolor=black,
pdfpagemode=none,
pdftitle={Curriculum Vitae},
pdfauthor={David Shaner LeBauer},
pdfsubject={Curriculum Vitae},
pdfkeywords={Biogeochemistry, Fungi, mushroom, nitrogen, carbon, global change, earth system science,}
}
%%===========================================================================%%
%---------------------------------------------------------------------------
% Document Specific Customizations
% Make lists without bullets and with no indentation
\setlength{\leftmargini}{0em}
\renewcommand{\labelitemi}{}
% Use large bold font for printed name at top of pages
\renewcommand{\namefont}{\large\textbf}
%---------------------------------------------------------------------------
\name{David LeBauer}
\begin{document}

\begin{resume}
\begin{ncolumn}{2}
Energy Biosciences Institute & Phone: (919) 275-0360\\
Institute for Genomic Biology &  {\tt dlebauer@illinois.edu}\\
University of Illinois at Urbana-Champaign & \\
\end{ncolumn}
%---------------------------------------------------------------------------
\vspace{-0.6cm}
\section{Education}\vspace{0.1cm}
\begin{format}
\title{l}\dates{r}\\
\employer{l}\location{r}\\
\end{format}
\title{Ph.D.} \dates{2004--2008} \employer{Department of Earth System Science} \location{University of California at Irvine}
\begin{position} \end{position} \vspace{-0.6cm}

\title{M.S.} \dates{2001--2003} \employer{Ecology Graduate Group, Agriculture emphasis} \location{University of California at Davis}
\begin{position} \end{position} \vspace{-0.6cm}

\title{B.S. \textit{Honors}} \dates{1994--1998} \employer{Biology, Ecology concentration} \location{Duke University}
\begin{position} \end{position} \vspace{-0.7cm}
%---------------------------------------------------------------------------
\section{Research Experience}\vspace{0.1cm}
\begin{format}
\title{l}\dates{r}\\
\employer{l}\location{r}\\
\end{format}

\title{Postdoctoral Research Associate}
\employer{Dr.Mike Dietze}
\location{University of Illinois at Urbana-Champaign}
\dates{2009-present}
\begin{position}
I am utilizing an ecophysiological model to understand and predict the implications of different potential biofuel crop species for regional and global carbon, nitrogen, and water cycling.
\end{position}
\title{PhD. Candidate}
\employer{Dr. Kathleen Treseder}
\location{University of California at Irvine}
\dates{2004-2008}
\begin{position}
Investigated the impact of nitrogen on global plant production, net ecosystem carbon balance, and the effects of biodiveristy on decomposition.
\end{position}
\title{Laboratory and Operations Manager}
\employer{Dr. Omon Isikhuemhen}
\location{N.C. A\&T State University}
\dates{2003--2004}
\begin{position} Managed research program focused on agricultural, industrial, and medicinal applications of white-rot fungi, advised student members of the lab, coordinated extension activities.
\end{position}
\title{Research Assistant}
\employer{Dr. Louise Jackson and Dr. Kate Scow}
\location{University of California at Davis}
\dates{2001--2003}
\begin{position} Investigated the effect of mycorrhizal fungi and N addition on the C, N, and P content of organic canning tomatoes, and the response of nitrifier bacteria communities to N in agricultural fields.
\end{position}
\title{Visiting Researcher}
\employer{Dr. Omon Isikhuemhen}
\location{Mikrobiologizky Ustav, Prague}
\dates{1998}
\begin{position}
Investigated ligninolytic enzyme production in white rot fungi, cultivation of \textit{Pleurotis tuberregium}.
\end{position}\vspace{-0.2cm}
\title{Independent Research}
\employer{Dr. Rytas Vilgalys}
\location{Duke University}
\dates{1998--2001}
\begin{position}
Establishment of a landscaped mushroom garden in the Sarah P. Duke Gardens, isolation and cultivation of local fungi, teaching.
\end{position}
\title{Independent Research}
\employer{Dr. Janis Antonovics}
\location{Duke University}
\dates{1997--1998}
\begin{position}
Honors Thesis: "Mycorrhizally mediated competition between two host plants"
\end{position}
\title{Labratory and Field Technician}
\employer{Dr. Jacqueline Mohan\\ Dr. James Clark and Dr. William Schlesinger}
\location{Duke University}
\dates{1997--2001}
\begin{position} Research assistant, plant community response to free-air CO$_2$ enrichment (Duke-FACE), effects of the environment on seedling establishment and recruitment.
\end{position}
%%---------------------------------------------------------------------------%%
\newpage
\opening

\section{Teaching Experience}\vspace{0.1cm}
\begin{format}
\title{l}\dates{r}\\
\employer{l}\location{r}\\
\end{format}

\title{Teaching Assistant}
	\employer{BIO 100LW: Experimental Biology Laboratory}
	\location{UC Irvine}
	\dates{Fall 2008}
	\begin{position}\end{position} \vspace{-0.5cm}

\title{Teaching Assistant / Instructor}
	\employer{CA State Summer School for Math and Science}
	\location{UC Irvine}
	\dates{Summer 2008 / 2009}
	\begin{position} \end{position} \vspace{-0.5cm}

\title{Teaching Assistant}
	\employer{ESS 134: GIS for Environmental Science}
	\location{UC Irvine}
	\dates{Winter 2006, 2008}
	\begin{position} \end{position} \vspace{-0.5cm}

\title{Teaching Assistant}
 \employer{ESS 13: Global Change Biology}
 \location{UC Irvine}
 \dates{Spring 2007}
 \begin{position} \end{position} \vspace{-0.5cm}

\title{Agricultural Extension Instructor}
	\employer{Log Cultivation of Shiitake Mushrooms}
	\location{NC A\& T SU}
	\dates{Fall 2003- Spring 2004}
	\begin{position} \end{position} \vspace{-0.5cm}

\title{Teaching Assistant}
	\employer{PLP 140: Mushroom Cultivation}
	\dates{Winter 2003}
	\location{UC Davis}
	\begin{position} \end{position} \vspace{-0.5cm}

\title{Co-leader}
	\employer{HsC 79.17: Project WILD Outdoor Experiential Education Course}
	\dates{Spring 1996}
	\location{Duke University}
	\begin{position} \end{position}\vspace{-0.25cm}

%%---------------------------------------------------------------------------%%
\section{Funding}\vspace{0.1cm}
\begin{format}
\title{l}\dates{r}\\
\employer{l}\location{r}\\
\end{format}

\title{Graduate Student Fellowship} 
	\dates{2005-2007}
	\employer{Kearney Soil Science Foundation, Davis, CA.}
	\location{\$33,580}
	\begin{position} \end{position}\vspace{-0.5cm}

\title{Mildred E. Mathias Graduate Research Grant} 
	\employer{UC Natural Reserve System}
	\dates{2006}
	\location{\$1000}
	\begin{position} \end{position}\vspace{-0.5cm}

\title{Departmental Fellowship} 
	\employer{Earth System Science, UC Irvine} 
	\dates{2004-2005}
	\location{\$32,000}
	\begin{position} \end{position}\vspace{-0.5cm}

\title{Benenson Award in the Arts}
	\employer{Duke University} \location{\$1500}
	\dates{1998}
	\begin{position} \end{position}\vspace{-0.5cm}

%-------
\section{Publications}\vspace{0.5cm}
\begin{itemize}
\item{\bf{Journal Articles}}
\item LeBauer, D.S. Litter degradation rate and beta-glucosidase increase with fungal diversity. 2010. Canadian Journal of Forest Research, \textit{in press}.
\item Wang, D., D.S. LeBauer and M.C. Dietze. 2010. A quantitative review comparing the yield of switchgrass in monocultures and mixtures in relation to climate and management factors. Global Change Biology Bioenergy. \textit{in press}
\item Allison, S.D, D.S. LeBauer, M.R. Ofrecio, R. Reyes, A.M. Ta, and T.M. Tran. 2008. Low levels of nitrogen addition stimulate decomposition by boreal forest fungi. Soil Bio. Biochem 41:293-302. 
\item LeBauer, D.S. and K.K. Treseder. 2008. Nitrogen Limitation of Net Primary Productivity in Terrestrial Ecosystems is Globally Distributed. Ecology 89 (2): 371-379
\item Okano, Y., K.R. Hirstova, C.M. Leutenegger, L.E. Jackson, R.F. Denison, B. Gebreyesus, D.S. LeBauer, and K.M. Scow. (2004) Effects of ammonium on the population size of ammonia-oxidizing bacteria in soil- Application of real-time PCR. Applied and Environmental Microbiology 70:1008-1016.

%\item{\bf{Thesis chapter}} \textit{in prep}
%\item LeBauer, D.S., E.W. Seabloom, and W.S Harpole. Ecosystem responses to a gradient of nitrogen fertilization in Southern California grassland.

\newpage \opening
\section{Other Publications}%\vspace{0.5cm}
\item{\textbf{Book Chapter}}
\item Isikhuemhen, O.S. and D.S. LeBauer. 2004. Growing \textit{Pleurotus tuberregium}. p270-281 in Mushroom Grower's Handbook 1: Oyster Mushroom Cultivation. MushWorld. Seoul, Korea.
\item{\textbf{Technical Document}}
\item LeBauer, D.S., D. Wang, X. Feng, and M.C. Dietze. 2010. The EBI Biofuel Crop Traits Database: Database Description and Users Guide ver 0.1
\item{\textbf{News Articles}}
\item LeBauer, D.S. 2008. Nitrogen Pollution Boosts Plant Growth. Scitizen.com, France
\item LeBauer, D.S. 2004. Consider Shiitake Cultivation! Stewardship News, Carolina Farm Stewardship Association. Pittsboro, NC. 24(2):8-11.
\end{itemize}
%----------------------
\section{Selected Presentations}\vspace{0.5cm} 
\begin{itemize}
\item LeBauer, D.S., and K.K. Treseder 2007. Nitrogen deposition reduces nitrogen limitation of terrestrial net primary production, ESA, San Jose, CA
\item LeBauer, D.S., and K.K. Treseder 2006. Nitrogen Limitation of Terrestrial Net Primary Production: Global Patterns From Field Studies with Nitrogen Fertilization, AGU, San Francisco, CA
\item LeBauer, D.S., and K.K. Treseder 2006. Grassland carbon dynamics along nitrogen and depth gradients. ESA, Memphis, TN
\item LeBauer, D.S. and O.S. Isikhuemhen. 2003. Small farm mushroom production for food and medicine. (Workshop) CFSA Sustainable Agriculture Conference, Rock Hill, SC
\end{itemize}
%%---------------------------------------------------------------------------%%
\section{Mycological Consulting}\vspace{0.5cm}
\begin{itemize}
\item{Cultivation,} Oyster mushroom (\textit{Pleurotus ostreatus}) on tomato (\textit{Lycopersicon esculentum}) pomace, for Dr. Annie King and Dr. Jamal Assi, UC Davis 2003
\item{Cultivation and Education,} Crabtree Valley Farms, Chatanooga, TN. 2000; Wildflower Organics, Dawsonville, Ga., 1999-2000; Sustenance Farm, Bear Creek, NC., 1999
\item{Mycological Landscaping,} Sarah P. Duke Gardens, Durham, NC. Summer 1998, 1999
\end{itemize}
%------------------------------------------------------------------------------
\vspace{-0.25cm}
\section{Professional Affiliations}\vspace{0.5cm}
\begin{itemize}
\item{Ecological and Mycological Societies of America, American Geophysical Union, Sigma Xi}
\end{itemize}

%---------------------------------------------------------------------------g
\vspace{-0.25cm}
\section{Service}
\begin{itemize}
\item Graduate Student Representative, Department of Earth System Science, University of California at Irvine, 2006-2007
\item Judge, California State Science Fair, 2006-2009
\item Ad hoc reviewer: Geophysical Research Letters, Ecology, Plant and Soil
\item Graduate Student Seminar Organizer, Earth System Science, University of California at Irvine, 2005-2006
\item Mentor, FARMS Leadership, Inc. 2003. 8th grade science project to investigate soil quality and mycorrhizal colonization levels in gardens. Sacramento, CA.
\item Guest Teacher, K-12, 1998-99. Scientists 
in the Classroom, Sigma Xi, Durham, NC.
\end{itemize}
%%%RESEARCH INTERESTS: COMMENTED OUT FOR NOW%%%
%\vspace{-0.25cm}
%\section{Research Interests}\vspace{0.5cm}
%\begin{itemize}
%\item{Biogeochemistry} understanding ecosystem functions as observed in transformations, flows, and fates of nutrients; particular focus on the role of fungi and microorganisms in nutrient cycling and soil function.
%\item{Plant-mycorrhizal interactions} Nutrient cycling in natural and managed ecosystems; evolution of symbiosis; models of plant-fungal resource trade.
%\item{Decomposer fungi} Application of fungi to sustainability through conversion of ligno-cellulosic waste into food and medicine and through Stametsian bioremediation.
%\end{itemize}
%\section{Additional Coursework}
%\begin{itemize}
%\item Plant Functional Traits, NCEAS Distributed Graduate Seminar, 2008
%\item Probability, Bayesian Statistics, UCI Statistics Dept, 2007;
%\item Sustainable Agriculture, Central Carolina Community College, 1999 
%\end{itemize}
%%---------------------------------------------------------------------------%%
\end{resume}
\end{document}
%%===========================================================================%%
