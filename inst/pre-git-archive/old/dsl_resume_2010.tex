% $Id$ -*-LaTeX-*-

% Purpose: Full Curriculum Vitae
% URL: 
% http://dust.ess.uci.edu/job/cv/cv.pdf
% http://dust.ess.uci.edu/job/cv/cv.html

% Usage (see also end of file):
% cd ~/job;make -W cv.tex cv.pdf;cd -

\documentclass[11pt,twoside]{article}

% Standard packages
\usepackage{ifpdf} % Define \ifpdf
\ifpdf % PDFLaTeX
\usepackage[pdftex]{graphicx} % Defines \includegraphics*
\pdfcompresslevel=9
\usepackage{thumbpdf} % Generate thumbnails
\else % !PDFLaTeX
\usepackage{graphicx} % Defines \includegraphics*
\fi % !PDFLaTeX
%\usepackage{bibunits} % Multiple bibliographies in single article
\usepackage{array} % Table and array extensions, e.g., column formatting
\usepackage[dayofweek]{datetime} % \xxivtime, \ordinal
\usepackage{mdwlist} % Compact list formats \itemize*, \enumerate*
\usepackage{revnum} % Reverse enumerate
\usepackage{times} % Postscript Times-Roman font KoD99 p. 375
\usepackage{url} % Typeset URLs and e-mail addresses

% hyperref is last package since it redefines other packages' commands
% hyperref options, assumed true unless =false is specified:
% backref       List citing sections after bibliography entries
% baseurl       Make all URLs in document relative to this
% bookmarksopen Unknown
% breaklinks    Wrap links onto newlines
% colorlinks    Use colored text for links, not boxes
% hyperindex    Link index to text
% plainpages=false Suppress warnings caused by duplicate page numbers
% pdftex        Conform to pdftex conventions
% Colors used when colorlinks=true:
% linkcolor     Color for normal internal links
% anchorcolor   Color for anchor text
% citecolor     Color for bibliographic citations in text
% filecolor     Color for URLs which open local files
% menucolor     Color for Acrobat menu items
% pagecolor     Color for links to other pages
% urlcolor      Color for linked URLs
\ifpdf % PDFLaTeX
\usepackage[backref,breaklinks,colorlinks,citecolor=blue,linkcolor=blue,urlcolor=blue,hyperindex,plainpages=false,pdftex]{hyperref} % Hyper-references
\usepackage{fancyhdr}
\pdfcompresslevel=9
\else % !PDFLaTeX
\usepackage[backref=false,breaklinks,colorlinks=false,hyperindex,plainpages=false]{hyperref} % Hyper-references
\fi % !PDFLaTeX
\hypersetup{
letterpaper,
colorlinks,
urlcolor=black,
pdfpagemode=none,
pdftitle={Curriculum Vitae},
pdfauthor={David Shaner LeBauer},
pdfsubject={Curriculum Vitae},
pdfkeywords={Biogeochemistry, Fungi, mushroom, nitrogen, carbon, global change, earth system science,}
}
% Personal packages
\usepackage{csz} % Library of personal definitions
\usepackage{abc} % Alphabet as three letter macros
\usepackage{chm} % Commands generic to chemistry
\usepackage{hyp} % Hyphenation exception list

% Margins
%\topmargin 0in \headheight 0pt \headsep 0pt % No headers
\topmargin -0.25in \headheight 0.125in \headsep 0.125in % Margin compensates for any headers so that body can still be nine inches
\textheight 9in \textwidth 6.5in
\oddsidemargin 0in \evensidemargin 0in
\marginparwidth 0pt \marginparsep 0pt
%\footskip 0pt
\footskip 0.25in
\footnotesep=0pt

% Commands which must be executed in preamble
% \DeclareFixedFont{command}{code}{family}{series}{shape}{size}
\DeclareFixedFont{\xiirn}{OT1}{cmr}{m}{n}{12}
\DeclareFixedFont{\xiitiit}{OT1}{cmti}{m}{it}{12}
\makecompactlist{revnumerate*}{revnumerate}

% Horizontal space 
\setlength{\parindent}{0mm}

\begin{document}

\pagenumbering{arabic}
\setcounter{page}{1}
\pagestyle{fancy}
\lhead{\textsc{David~S. LeBauer}}
\chead{}
\rhead{\thepage}
\lfoot{}
\cfoot{}
\rfoot{}
\newcommand{\dslnote}{\csznote}
%\pagestyle{empty}
%\pagestyle{plain}
\thispagestyle{plain}

\begin{center}
%{\small Web: \url{https:///netfiles.uiuc.edu/dlebauer/www/cv.pdf} \hfill Updated: \shortdate\today}\\
%{\hfill Updated: \shortdate\today}\\
{\textit{Resume}}\\
%{\xiirn \scshape CHARLES~S. (CHARLIE) ZENDER}\\
{\large\textsc{David~S. LeBauer}}\\
\end{center}

Energy Biosciences Institute \hfill Voice: (919)\thinspace 275-0360\\
University of Illinois \hfill Fax: (217)\thinspace 244-3637  \\
Urbana, IL~~61801 \hfill Email: dlebauer@illinois.edu\\

\medskip
\textbf{EMPLOYMENT}
\vspace{-0.5em}

\begin{table}[h]
\begin{tabular}{ >{\raggedright}p{18.0em}<{} >{\raggedright}p{20.0em}<{}l}
 Postdoctoral Research Associate& University of Illinois, Urbana-Champaign & 2009-- present\\ 
 Graduate Research \& Teaching Assistant& University of California at Irvine&2004--2008  \\ 
 Laboratory Manager and Extension Educator& University of North Carolina A\&T &2003--2004 \\
 Graduate Research \& Teaching Assistant& University of California at Davis  &2001--2003\\
 Laboratory and Field Technician& Duke University &1997--2001 \\
\end{tabular}
\end{table}

\vspace{-0.5em}

\textbf{EDUCATION}
\vspace{-0.5em}

\begin{table}[h]
\begin{tabular}{ >{\raggedright}p{2.0em}<{}  >{\raggedright}p{15.0em}<{} >{\raggedright}p{20.0em}<{}   l } 
Ph.D. & Earth System Science & University of California at Irvine & 2008\\
M.S. & Agricultural Ecology & University of California at Davis & 2003 \\
B.S. &Biology &  Duke University & 1998 \\
\end{tabular}
\end{table}
\vspace{-0.5em}

\textbf{TRAINING}% AND ADDITIONAL COURSEWORK}
\vspace{-0.5em}

\begin{table}[h]
\begin{tabular}{  >{\raggedright}p{18.0em}<{}  >{\raggedright}p{20.0em}<{}    l } 
 Flux Measurement and Modeling& University of Colorodo & 2010  \\
 Plant Functional Traits& Nat'l Ctr for Ecological Analysis and Synthesis& 2008\\
 Radiocarbon in Earth System Science& University of California at Irvine& 2004\\
 Sustainable Agriculture& Central Carolina Community College&1999\\ 
\end{tabular}
\end{table}
\vspace{-0.5em}
\textbf{TEACHING}
\vspace{-0.5em}
\begin{table}[h!]
\begin{tabular}{ >{\raggedright}p{18.0em}<{}  >{\raggedright}p{20.0em}<{} l } 
 Global Change Biology& California Summer School for Science and Math&2008,2009\\
 Experimental Biology Laboratory& University of California at Irvine &2008\\
 GIS for Environmental Science& University of California at Irvine& 2006, 2008\\
 Global Change Biology& University of California at Irvine& 2007\\
 Archaeology and Geology& Duke University & 1997\\
\end{tabular}
\end{table}
\vspace{-0.5em}

\textbf{SKILLS}
\vspace{0.5em}

\textbf{General:}  Mentoring, Management, Writing, Statistics, Data Visualization;
\textbf{Lab/Field:}  Plant and Soil Chemistry, Isotopic Analysis, GC/MS, Flux Measurements (Eddy Covariance, Photosynthesis, Respiration), Microscopy, Spectroscopy, Enzyme Analyses, Sterile Culture, Plant and Fungal Identification;
\textbf{Computer:} Linux, Microsoft, emacs, R, BUGS, SQL, bash, Redmine, Office, \LaTeX

\dslnote{
\textbf{TEACHING AND RESEARCH OVERVIEW}
\par\medskip
%for NEON Biometeorologist position 20100810
My research focuses on the response of terrestrial biogeochemistry to anthropogenic change, including the indirect eutrophication of the biosphere and the direct management of land for food and fuel production. My goal is to develop scientific assessments that promote integration of science into policy and management decision making. 
\dslnote{%left out of NEON statement 
For example,  the  develop tools to improve understanding and prediction of ecosystem functioning, with the aim ofmore ecologically informed approaches to agriculture, forestry, and urban development. } % end dslnote   
To accomplish this
goal, I combine model development with data assimilation to refine and improve the accuracy of ecological
forecasts. This work benefits from collaboration with agronomists, atmospheric scientists, hydrologists,
statisticians, and computer scientists. On a global scale, my goal is to predict global consequences of anthropogenic
change. On a local and regional scale, my primary objective is to develop models that farmers,
foresters, and urban planners can use to predict the ecological consequences of local and regional managment
decisions. Furthermore, I value and enjoy the opportunity to educate students. My approach to science
education is to engage students in the process of scientific discovery. One objective of this approach is to
enable students to collect data that will become long term records of ecosystem functioning. These records
could be used to test and parametrize predictive models. By engaging participants from a diversity of fields
and settings, it will be possible to promote scientific literacy outside of academia and thereby foster the use
of science at all levels of learning and decision making.
\dslnote{%CSZ's statement
I educate, train students, and conduct research at UC Irvine.
My teaching includes undergraduate and graduate courses on climate and 
physics, and graduate seminars on physical climate modeling.
Our research group studies the energy and trace species that pass
through Earth's atmosphere.    
We model the microphysics of trace gas, aerosol, and surface
interactions with Earth's radiative, thermodynamic, and chemical
budgets.    
We then (often) parameterize these effects in climate models.
The model simulations, combined with lab, field, and satellite data,
help us predict and attribute features of climate and climate change.
Current research includes mineral dust, meteoric, and carbonaceous 
aerosols, snow lifecycle and albedo, aerosol impacts on ocean
biogeochemistry, wind-driven surface energy/mass exchange,
climate-disease links, and terascale data analysis. 
Our aerosol generation, radiative transfer, and data processing models
are freely available and are used in geoscience research institutions
world-wide.
} % end dslnote
\par\bigskip
} % end dslnote
\csznote{
\textbf{PROFESSIONAL INTERESTS}
\par\medskip
\dslnote{ % CSZ's professional interests
My research focuses on two complementary areas: terrigenic mineral
dust aerosol and the disposition of solar radiation in the climate
system. 
Currently I am examining Earth's radiative and photochemical responses
to mineral dust aerosol.   
Related interests are the roles of dust particles in air quality, 
ice crystal nucleation, paleoclimate, and fertilization of remote
ecosystems.     
My studies combine satellite, field, and laboratory observations of
aerosols, clouds, trace gases, soil types, and surface properties with 
radiative transfer and large scale climate system models.  
The ultimate goal of these studies is to improve our understanding
andprediction of climate and climate change.   
\par\medskip

\begin{itemize*}
\item[] Mineral dust aerosol---its distribution, source and sink mechanisms,
direct and indirect radiative forcing, and impact on tropospheric
processes.

\item[] Radiation and climate---closing the surface shortwave energy
budget, using field observations to investigate causal mechanisms of
excess shortwave absorption, using GCMs to study the effect of
aerosols and clouds on climate.
\end{itemize*}

% Expertise keywords sent 20091219 to Round Table Group (RTG)
Climate Change, Physics, Meteorology, Supercomputing, Pollution,
Arctic, Snow, Dust, Black Carbon, Light

% Profile sent 20090806 to Jerry Schubel for LBAOP profile
% Profile sent 20091219 to Round Table Group (RTG)
Charlie Zender is an atmospheric physicist and educator at UC Irvine.
He has led the Climate, Health, Aerosols, Radiation, and Microphysics
(CHARM) group in their Department of Earth System Science since 1999.
CHARM studies the distribution and fluxes of energy and trace species
that interact with Earth's atmosphere. Their current research includes
mineral dust and carbonaceous aerosols, snow/firn evolution, aerosol
impacts on ocean biogeochemistry, wind-driven energy exchange, and
terascale data analysis.  Zender actively peer-reviews mansuscripts
and proposals for professional organizations and agencies (NSF, NASA)
and journals. His governmental service includes the California Climate
Change Advisory Committee, and his testimony before Congress on the
effects of aerosols on Arctic climate.  Zender also consults pro bono
for several non-governmental organizations dedicated to clean air and
healthy environments.  Prior to joining UCI in 1999, Zender studied at
Harvard (AB 1990), the University of Colorado, Boulder (PhD 1996), and
the National Center for Atmospheric Research (postdoc).

% Profile (30-words) sent 20100205 to Corinne Kisner for Asilomar
My group studies dust, snow, fire, wind, and their interactions with 
climate and health. I also teach undergraduates about geoengineering.
We discuss scientific, technical, economic, and ethical issues.
It's provocative!

% Profile sent 20100316 to Edward Patte for GAFOS profile
I studied physics and astronomy at Harvard, before finding my true
passion for climate and atmospheric physics in Boulder where I
completed my doctoral work (University of Colorado) and stayed for a 
postdoc at the National Center for Atmospheric Research.
I joined the Department of Earth System Science at UC Irvine in 1999.
What most fascinates me is the effect of previously neglected physical
processes on climate. My Climate, Health, Aerosols, Radiation, and
Microphysics (CHARM) group studies the distribution and fluxes of
energy and trace species that interact with Earth's atmosphere. 
Our current research includes desert dust and carbonaceous aerosols,
snow/firn evolution, aerosol impacts on ocean biogeochemistry,
wind-driven energy exchange, and terascale data analysis. 
I try to give back to society by serving government, e.g., the
California Climate Change Advisory Committee, and giving testimony
before Congress on the effects of aerosols on Arctic climate.  
I also consult pro bono for several non-governmental organizations
dedicated to clean air and healthy environments. Recently I have
become interested in geoengineering methods to cool Earth.

% Profile sent 20090427 to Edward Patte for GAFOS profile
% Profile sent 20090511 to Linda Brown for LBAOP profile
% Profile sent 20100121 to Linda Cain for SACC profile
Professor Zender is an atmospheric physicist and educator.  
He leads the Climate, Health, Aerosols, Radiation, and Microphysics
(CHARM) group in the Department of Earth System Science at UC Irvine.
They study the distribution and fluxes of energy and trace species
in Earth's atmosphere.  Their current research includes desert dust
and combustion aerosols, snow/firn evolution, particulate impacts on
ocean nutrient cycling, wind-driven energy exchange, and terascale
data analysis.  Prior to joining UCI in 1999, Zender studied at
Harvard (AB 1990), the University of Colorado, Boulder (PhD 1996), and
the National Center for Atmospheric Research (postdoc).

% Profile sent 20071013 to Gary Robbins for OCR profile
% Profile sent 20090311 to Isabella Velicogna for UC CIRSI proposal
Research: I study the roles of clouds, gases, and aerosols in
the global energy budget and in climate.
My group has active research in snow physics, desert dust, polar
climate, biomass burning-El Nino interactions, and air-surface
exchange, and large scale data analysis.   
We often construct detailed microphysical computer simulations of
complex environmental processes (e.g., dust storms, snow aging). 
To improve these predictions they use measurements from 
remote meteorological stations, ocean buoys, aircraft, satellites, and
laboratory and in~situ instruments. 
These results feed into the climate models that estimate the global
effects of clouds, dust, soot, and wind in past, present, and future
climates.       
We also create software used world-wide to efficiently analyze the
large datasets produced by supercomputer simulations and satellite
observations.  
Noteworthy: I participate in International Polar Year (IPY) activities
and he recently testified to Congress on the causes of Arctic warming.   
} % end dslnote
} % end csznote

%\clearpage % fxm need better way to keep title and list on same page

\csznote{ % 20100520: How to generate CV TOC?
% Option 1: Redefine \section 
\newcommand{\tmpsection}[1]{}
\let\tmpsection=\section
\renewcommand{\section}[1]{\tmpsection{\underline{#1}}}
\section{Manuscripts in Peer-Review}\label{sxn:mpr}

% Option 2: Manually add sections to TOC
\addtocontents{toc}{MANUSCRIPTS IN PEER-REVIEW}

% Required: Following line generates TOC:
\tableofcontents
} % end csznote on CV TOC

\dslnote{ %no manuscripts in peer review
\textbf{MANUSCRIPTS IN PEER-REVIEW}

} % end dslnote
\newpage
\textbf{SOFTWARE}
\vspace{-0.5em}
\begin{itemize*}
\item  \emph{Lead developer}, Biofuel Ecophysiological Traits and Yields Database (BETYdb): Database of plant traits and yield data, used to synthesize previous and ongoing research, to support model parameterization and validation, and to provide a central data clearinghouse for biofuel research. \\\texttt{ebi-forecast.igb.uiuc.edu/bety}
\item  \emph{Lead developer}, Predictive Ecophysiological Analyzer (PEcAn): analytical toolbox and workflow to support management of model parameterization, execution, and analysis.  \\\texttt{www.pecanproject.org}
\end{itemize*}
\textbf{PEER-REVIEWED JOURNAL ARTICLES}
\vspace{-0.5em}
\begin{itemize*}
%\renewcommand{\labelenumi}{PJ\arabic{enumi}.} % KoD99 p. 70
% Begin Refereed articles
% Update protocol: Add here upon transition to "In Press"
% Reverse chronological in order of acceptance, i.e., transition to "In Press"
% This is because accepted publications should be on academic addenda
% Final publication dates may or may not be chronological
% This depends on lag between acceptance and publication
% fxm: Find DOIs on publications 1--7.
% Articles not retrieved by "Zender, CS" search on Thompson ISI Web of Knowledge:
% http://www.isiwebofknowledge.com
% HMZ08 (under "Zender, C" not "Zender, CS"), ZMT04 (no Eos)
% Journal Impact Factors (IFs) from Journal Citation Reports (JCR):
% http://admin-apps.isiknowledge.com/JCR/JCR

% EST: Environ. Sci. Technol. = Environmental Science & Technology
% GBC: 4.33 (2007)
% GMD: fxm (2009) % First issue in 2008, not ranked yet
% GRL: 2.74 (2007)
% JGR: 3.08 (2009)
% Science: 28.3 (2007)


\item \noindent% dlebauer2011fbm
\textbf{LeBauer, D.~S.}, C. Davidson, D. Wang, and M.~C. Dietze 201-. Facilitating feedbacks between measurements and models with a scientific workflow, the Predictive Ecosystem Analyzer (PEcAn). \textit{in review}
%\ifpdf % PDFLaTeX
%(\href{}{PDF})
%\fi % !PDFLaTeX

\item \noindent% dlebauer2010ldr
\textbf{LeBauer, D.~S.} 2010. Litter degradation rate and beta-glucosidase activity increase with fungal diversity. \textit{Can.\ J.\ For.\ Res.}, 40(6): 1076–-1085
\ifpdf % PDFLaTeX
(\href{https://netfiles.uiuc.edu/dlebauer/www/lebauer2010ldr.pdf}{PDF})
\fi % !PDFLaTeX

\item \noindent% wang2010qrc
Wang, D., \textbf{D.~S. LeBauer} and M.~C. Dietze. 2010. A quantitative review comparing the yield of
switchgrass in monocultures and mixtures in relation to climate and management factors. \textit{Glob.
Change Biol. Bioenergy}, 2(1):16--25
\ifpdf % PDFLaTeX
(\href{https://netfiles.uiuc.edu/dlebauer/www/wang2010qrc.pdf}{PDF})
\fi % !PDFLaTeX

\item \noindent% allison2009qrc
Allison, S.~D, \textbf{D.~S. LeBauer}, M.~R. Ofrecio, R. Reyes, A.~M. Ta, and T.~M. Tran. 2008. Low levels of
nitrogen addition stimulate decomposition by boreal forest fungi. \textit{Soil Bio. Biochem.}, 41:293--302
\ifpdf % PDFLaTeX
(\href{https://netfiles.uiuc.edu/dlebauer/www/allison2009qrc.pdf}{PDF})
\fi % !PDFLaTeX

\item \noindent% lebauer2008nln
\textbf{LeBauer, D.~S.} and K.~K. Treseder. 2008. Nitrogen Limitation of Net Primary Productivity in
Terrestrial Ecosystems is Globally Distributed. \textit{Ecology}, 89(2): 371--379
\ifpdf % PDFLaTeX
(\href{https://://netfiles.uiuc.edu/dlebauer/www/lebauer2008nln.pdf}{PDF})
\fi % !PDFLaTeX

\item \noindent% 
Okano, Y., K.~R. Hirstova, C.~M. Leutenegger, L.~E. Jackson, R.~F. Denison, B. Gebreyesus, \textbf{D.~S.
LeBauer}, and K.~M. Scow. 2004. Effects of ammonium on the population size of ammonia-oxidizing
bacteria in soil- Application of real-time PCR. \textit{Appl. Environ. Microb.}, 70:1008--1016
\ifpdf % PDFLaTeX
(\href{https://://netfiles.uiuc.edu/dlebauer/www/okano2004art.pdf}{PDF})
\fi % !PDFLaTeX
\end{itemize*}
% end REFEREED PUBLICATIONS

\textbf{BOOK CHAPTER}
\vspace{-0.5em}
\begin{itemize*}
\item \noindent 
Isikhuemhen, O.~S. and \textbf{D.~S. LeBauer} 2004. Growing Pleurotus tuberregium. p270-281 in Mushroom
Grower's Handbook 1: Oyster Mushroom Cultivation. MushWorld. Seoul, Korea.
\ifpdf % PDFLaTeX
(\href{https://netfiles.uiuc.edu/dlebauer/www/IsikhuemhenLeBauer2004GrowingPleurotustuberregium.pdf}{PDF}) 
\fi % !PDFLaTeX
\end{itemize*}


% end OTHER CONFERENCES ATTENDED

% fxm: Jeff Dozier's CV is an excellent example of organization, and
% http://www2.bren.ucsb.edu/%7Edozier/Dozier_CV.pdf
% includes instructive examples of distinguishing MS/PhD committee member/Chair
% Locating UCI theses:
% http://www.lib.uci.edu/libraries/collections/special/uci_td/locating.html

\textbf{FUNDING}
\vspace{-0.5em}
\begin{itemize*}
%\renewcommand{\labelenumi}{G\arabic{enumi}.} % KoD99 p. 70
\item Kearney Soil Science Foundation Graduate Student Fellowship:
  ``Grassland Carbon Dynamics along Nitrogen and Depth Gradients'',
  \$34000, 2005--2007 
\item NRS Mildred E. Mathias Graduate Student Research Grant Program:
  ``Decomposition responses to nitrogen in a California grassland'',
  \$1000, 2006 
\item UCI Earth System Science Departmental Fellowship,
\$32,000, 2004--2005
\item Benenson Award in the Arts,
  ``Construction of a landscaped mushroom garden''
\$1500, 1998
\end{itemize*}
% end SPONSORED RESEARCH PROJECTS

\textbf{MEMBERSHIPS}
\vspace{-0.5em}
\begin{itemize*}
%\renewcommand{\labelenumi}{M\arabic{enumi}.} % KoD99 p. 70
\item American Geophysical Union (AGU), 2005--present
\item Ecological Society of America (ESA), 2002--present
\item Sigma Xi, 2005--present
\end{itemize*}
% end MEMBERSHIPS

%\dslnote{ %can move consulting here
%\textbf{MYCOLOGICAL CONSULTING}
%\begin{itemize*}
%\item Cultivation of oyster mushroom (\textit{Pleurotus ostreatus}) on tomato (\textit{Lycopersicon esculentum}) %pomace, for Dr. Annie King and Dr. Jamal Assi, UC Davis 2003
%\item Cultivation and Education Crabtree Valley Farms, Chatanooga, TN. 2000; Wildflower Organics, Dawsonville, Ga%., 1999-2000; Sustenance Farm, Bear Creek, NC., 1999
%\item Mycological Landscaping, Sarah P. Duke Gardens, Durham, NC. Summer 1998, 1999
%%\renewcommand{\labelenumi}{C\arabic{enumi}.} % KoD99 p. 70
%\end{itemize*}
%% end CONSULTING ACTIVITIES
%} %end dslnote

\textbf{SERVICE}
\vspace{-0.5em}
\begin{itemize*}
%\renewcommand{\labelenumi}{SP\arabic{enumi}.} 
\item American Geophysical Union (AGU) Climate Science Experts Referral \href{http://www.agu.org/news/press/pr_archives/2010/2010-14.shtml}{Service}, December, 2009--present.
\item Reviewer, Ecology, Ecology Letters, Geophysical Research Letters,  Plant Soil, Applied Soil Ecology % need to maintain official count
\item Graduate Student Representative, Dept. of Earth System Science, UCI, 2006--2007
\item Graduate Student Seminar Organizer, Earth System Science, University of California at Irvine, 2005-2006
\end{itemize*}
% end SERVICE TO PROFESSIONAL SOCIETIES AND CONFERENCE ORGANIZATION


%\pagebreak
\dslnote{ %default should be to include references, alt. could have Refs avail. upon request
\textbf{REFERENCES for David~S. LeBauer (available upon request)}
} %end dslnote
\dslnote{
\textbf{REFERENCES for David~S. LeBauer}
\begin{itemize*}

\item Professor~Susan~E. Trumbore\\
Director, Department of Biogeochemical Processes\\
Max-Planck Institute for Biogeochemistry\\
Hans-kn\"{o}ll-Str. 10\\
07745 Jena\\
Germany\\
Phone: +49 3641 576110, E-mail: trumbore@bgc-jena.mpg.de

\item Dr. Michael~C. Dietze\\
Department of Plant Biology\\
University of Illinois at Urbana-Champaign\\
126 IGB MC-116\\
Urbana,~IL~~61801-\\
Phone: (217) 333-6177, E-mail: mcdietze@illinois.edu

\item Dr.~W.~Stanley Harpole\\
Assistant Professor, Department of Ecology, Evolution, and Organismal Biology\\
Iowa State University\\
133 Bessey Hall\\
Ames,~IA~~50011 \\
Phone: (515) 294-7253, E-mail: harpole@iastate.edu

\item Dr.~Omoanghe.~S. Isikhuemhen\\
Associate Professor, \\
Department of Natural Resources and Environmental Design\\
North Carolina A\&T State University\\
Greensboro,~NC~~27411\\
Phone: (336) 334-7259, E-mail: omon@ncat.edu
\end{itemize*}
}


\csznote{ 
% Usage: Place usage here at end of file so comment character % not needed
cd ~/job;make -W cv.tex cv.dvi cv.ps cv.pdf cv.txt;cd -
scp ${HOME}/job/cv.dvi ${DATA}/ps/cv.pdf ${DATA}/ps/cv.ps ${HOME}/job/cv.tex ${HOME}/job/cv.txt dust.ess.uci.edu:/var/www/html/job/cv

# NB: latex2html works well on cv.tex
# latex2html -dir /var/www/html/job/cv cv.tex
# NB: tth chokes on cv.tex
# cd ${HOME}/job;tth -a -Lcv -p./:${TEXINPUTS}:${BIBINPUTS} < ${HOME}/job/cv.tex > cv.html
# scp cv.html dust.ess.uci.edu:/var/www/html/job/cv
# NB: tex4ht works well on cv.tex
cd ${HOME}/job;htlatex cv.tex
scp cv*.css cv*.html dust.ess.uci.edu:/var/www/html/job/cv
# NB: tex4moz works well on cv.tex
cd ${HOME}/job;/usr/share/tex4ht/mzlatex cv.tex
scp cv*.css cv*.html cv*.xml dust.ess.uci.edu:/var/www/html/job/cv
% $: re-balance syntax highlighting
} % end csznote on usage

\end{document}
