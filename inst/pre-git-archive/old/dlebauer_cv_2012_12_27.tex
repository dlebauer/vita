\documentclass{xetexCV}

\cvname{David Shaner LeBauer}
%\cvimage{Albert-Einstein-2.jpg}
\institution{University of Illinois at Urbana-Champaign}
\contactaddress{Energy Biosciences Institute\\
  1206 W. Gregory Drive\\
  Urbana, Illinois \texttt{08540}\\
  United States of America}
\phonenumber{(919)\thinspace 275-0360}
\faxnumber{(217)\thinspace 244-3637}
\email{dlebauer@illinois.edu}
%\website{http://www.ias.edu/spfeatures/einstein/}

\hyphenpenalty=10000


\setmainfont[Ligatures={Common}, Numbers={OldStyle}]{Fontin}
\setsansfont{Fontin Sans}

\usepackage{hyperref}

\begin{document}
\makecvtitle

\cvsection{Summary}

My objective is to improve our understanding of ecosystems so that we can make more useful predictions of carbon, water, and energy balance of ecosystems in the face of global change.

\cvsection{Current Appointment}
Research Manager. Ecosystem Modeling Program, Energy Biosciences Institute, University of Illinois.

\cvsection{Areas of Specialization}
biogeochemistry, ecoinformatics, simulation modeling

\cvsection{Appointments Held}
Postdoctoral Researcher, University of Illinois. Advisor: Mike Dietze \years{2008-2012} \\
Instructor, California State Summer School for Science and Math. \ years{2008,2009}\\
Graduate Researcher, University of California at Irvine. Advisor: Kathleen Treseder \years{2004-2008} \\
Teaching Assistant,  \href{https://sites.google.com/site/bio100lwlab14/}{Experimental Biology Laboratory}, University of California at Irvine. \years{2006,2007} \\
Teaching Assistant, GIS in Environmental Science, University of California at Irvine. \years{2006,2007} \\
Laboratory Manager, Extension Educator, University of North Carolina A\&T University  \years{2003-2004} \\
Graduate Researcher, University of California at Davis. Advisors: Louise Jackson and Kate Scow \years{2001-2003} \\
Teaching Assistant, Mushroom Cultivation, University of California at Davis. \years{2003} \\
Laboratory and Field Technician, Duke University. \years{1996-2001} \\

\cvsection{Education}
PhD in Earth System Science, University of California at Irvine \years{2004-2008} \\
MS in Agricultural Ecology, University of California at Davis \years{2001-2003} \\
BS in Biology, with Honors, Duke University \years{1994-1998}

\cvsection{Grants, Honors and Awards}

``Energy Biosciences Institute Feedstock and Ecosystem Service Modeling Program'' \$615,000 2013-2014 (lead-author, with S. Long, E. DeLucia, C. Bernacchi)\\
\vspace{1}
``Model-data synthesis and forecasting across the upper Midwest: Partitioning uncertainty and environmental heterogeneity in ecosystem carbon''   (co-author, with M. Dietze, A. Desai, and P. Bajscy) \\
NSF Advances in Bioinformatics Infrastructure: \$770,653 5/2011--04/2013\\

\vspace{1}
``Grassland Carbon Dynamics along Nitrogen and Depth Gradients'' \\
Kearney Soil Science Foundation Graduate Student Fellowship:  \$34,000, 5/2005--4/2007 \\ 

\vspace{1}
``Decomposition responses to nitrogen in a California grassland'' \\
NRS Mildred E. Mathias Graduate Student Research Grant Program: \$1000, 2006 \\

\vspace{1}
 UCI Earth System Science Departmental Fellowship: \$32,000, 2004--05

\vspace{1}
 UCI Ecology and Evolutionary Biology Departmental Fellowship: \$32,000, declined
\pagebreak

\cvsection{Publications}
\cvsubsection{Journal Articles}

Dietze, M.C., D.S.~LeBauer, and R. Kooper \textit{in press}. On improving the communication between models and data. Plant Cell \& Environment.
\medskip
%%
Wang, D., \textbf{D.S.~LeBauer}, and M.C.~Dietze MC , \textit{in press}. Predicted yields of short-rotation hybrid poplar (Populus spp.) for the contiguous US. Ecological Applications. doi: 
\medskip
%%
\textbf{LeBauer, D.~S.}, D.~Wang,, C.~Davidson, K.~Richter, and M.~C. Dietze \textit{in press}. Facilitating feedbacks between field measurements and ecosystem models. Ecological Monographs. doi:
\medskip
%%
\textbf{LeBauer, D.~S.} 2010. Litter degradation rate and beta-glucosidase activity increase with fungal diversity. \textit{Can.\ J.\ For.\ Res.}, 40(6): 1076–-1085 doi:
\medskip
%%
Wang, D., \textbf{D.~S. LeBauer} and M.~C. Dietze. 2010. A quantitative review comparing the yield of switchgrass in monocultures and mixtures in relation to climate and management factors. \textit{Glob.Change Biol. Bioenergy}, 2(1):16--25 doi:
\medskip
%%
Allison, S.~D, \textbf{D.~S. LeBauer}, M.~R. Ofrecio, R. Reyes, A.~M. Ta, and T.~M. Tran. 2008. Low levels of nitrogen addition stimulate decomposition by boreal forest fungi. \textit{Soil Bio. Biochem.}, 41:293--302

\medskip
%%
\textbf{LeBauer, D.~S.} and K.~K. Treseder. 2008. Nitrogen Limitation of Net Primary Productivity in Terrestrial Ecosystems is Globally Distributed. \textit{Ecology}, 89(2): 371--379

\medskip
%%
Okano, Y., K.R.~Hirstova, C.M.~Leutenegger, L.E.~Jackson, R.F.~Denison, B.~Gebreyesus, \textbf{D.S.~LeBauer}, and K.M.~Scow. 2004. Effects of ammonium on the population size of ammonia-oxidizing bacteria in soil- Application of real-time PCR. \textit{Appl. Environ. Microb.}, 70:1008--1016

\cvsubsection{Software}

PEcAn: The Predictive Ecosystem Analyzer\\
\hspace{}PEcAn is an ecoinformatics toolbox that synthesizes diverse data to evaluate and predict ecosystem carbon balance and biodiversity. PEcAn improves the ability to synthesize existing data and identifies additional data that would most effectively constrain uncertainty in ecological forecasts.


BETYdb: The Biofuel Ecophysiolotical Traits, Yields (and other ecosystem services) database\\
\hspace{}BETYdb provides an open access portal to facilitate organization and synthesis of information required to predict yields and other ecosystem functions provided by second generation biofuel ecosystems. BETYdb facilitates the management and synthesis of available data and provides the ecoinformatics infrastructure required by the PEcAn workflow. 



\cvsubsection{Newspaper Articles}

Einstein, Albert, et al. (December 4, 1948) \years{1940}, “To the editors", \emph{New York Times}\\
Einstein, Albert (May 1949) \years{1949}, “Why Socialism?", \emph{Monthly Review}.
\end{document}
