% LaTeX resume using res.cls
\documentclass[line]{res} 
%\usepackage{helvetica} % uses helvetica postscript font (download helvetica.sty)
\usepackage{newcent}   % uses new century schoolbook postscript font 

\begin{document}

\name{David S. LeBauer}
% \address used twice to have two lines of addresm

 
\begin{resume}

785 Harbor Cliff Way Unit 174 \hfill dlebauer@gmail.com\\
Oceanside, CA, 92054 \hfill (760) 722-1206\\

%\section{OBJECTIVE}       A position in the field of applied ecosystem science. 
 
\vspace{0.2in}
\section{OBJECTIVE} My objective is to develop and promote sustainable practicesthat have a strong foundation in ecosystem science. Humans can enhance ecosystem functions other than food production through management, and this presents valuable opportunities to farmers, foresters, and other land managers.   
\section{EXPERIENCE}
\subsection{Research}
\subsection{Consulting}
\subsection{Teaching}
\subsection{Education}

 {\sl PhD,} Earth System Science \hfill 2008 \\
                % \sl will be bold italic in New Century Schoolbook (or
	        % any postscript font) and just slanted in
		% Computer Modern (default) font
                \hfill University of California, Irvine, CA\\
                Thesis: Ecosystem carbon balance under nitrogen enrichment
                 
                {\sl MS,} Ecology, Agriculture emphasis \hfill 2003   \\
                % \sl will be bold italic in New Century Schoolbook (or
	        % any postscript font) and just slanted in
		% Computer Modern (default) font
                \hfill University of California, Davis, CA


                {\sl BS,} Biology, Ecology concentration \hfill 1998 \\
                \hfill Duke University, Durham, NC 
 
 
\section{RESEARCH} {\sl Graduate Student Researcher} \hfill  2004-2008\\
                 Dr. Kathleen Treseder \hfill UC Irvine
                 \begin{description}  \itemsep -2pt % reduce space between items
                 \item Investigated the impact of nitrogen on plant growth, 
                   net ecosystem carbon balance, 
                   and the effects of biodiversity on decomposition. 
                \end{description}
 
                {\sl Laboratory and Operations Manager} \hfill 2003-2004 \\
                Dr. Omon Isikhuemhen, Mushroom Biology and Biotechnology Lab \hfill NC A\&T SU 
                 \begin{description}  \itemsep -2pt %reduce space between items
                 \item Investigated agricultural, industrial, and medicinal 
                   applications of white-rot fungi
                  \item advised student members of lab
                  \item coordinated statewide spawn distribution
                    and farmer education. 
                 \end{description} 
                
                 {\sl Graduate Student Researcher} \hfill  2001-2003 \\
                 Drs. Louise Jackson and Kate Scow \hfill UC Davis
                  \begin{description}
                   \item Investigated the effect of mycorrhizal fungi and 
                    N addition on the C, N, and P content of organic 
                    canning tomatoes, 
                   \item Investigated the response of nitrifier bacteria 
                    communities to N in agricultural fields.
                   \end{description} 
 
                 {\sl Laboratory and Field Technician} \hfill  1997-2001 \\
                 Drs. Jacqueline Mohan, William Schlesinger, and Jim Clark \hfill Duke University
                  \begin{description}
                   \item Monitored seedling recruitment and growth under CO$_2$ enrichment
                   \item Mapped, identified, and measured trees
                   \item Collected data for species-specific allometric equations 
                 \end{description}

                {\sl Independent Research} \hfill  1998-2001 \\
                 Dr. Rytas Vilgalys, \hfill Duke University
                  \begin{description}
                   \item Established a landscaped mushroom garden
                   \item Isolated and cultivated fungi 
                   \item Taught mushroom cultivation
                   \end{description} 
                 
                 {\sl Independent Research} \hfill  1997-1998 \\
                 Dr. Janis Antonovics \hfill Duke University
                  \begin{description}
                   \item Honors Thesis: ``Mycorrhizally mediated competition between two host plants'' 
                   \end{description} 

\newpage
\opening
\section{TEACHING}

{\sl Teaching Assistant} \hfill                                       Fall 2008\\
BIO 100LW: Experimental Biology Laboratory \hfill               UC Irvine

{\sl Teaching Assistant / Instructor} \hfill               Summer 2008 / 2009\\
CA State Summer School for Math and Science  \hfill             UC Irvine

{\sl Teaching Assistant} \hfill                               Winter 2006, 2008\\
ESS 134: GIS for Environmental Science \hfill                   UC Irvine

{\sl Teaching Assistant} \hfill                                     Spring 2007\\
ESS 13: Global Change Biology  \hfill                           UC Irvine

{\sl Agricultural Extension Instructor} \hfill            Fall 2003- Spring 2004\\
Log Cultivation of Shiitake Mushrooms \hfill                 NC A\&T SU

{\sl Teaching Assistant} \hfill                                     Winter 2003\\
PLP 140: Mushroom Cultivation \hfill                             UC Davis

{\sl Co-leader} \hfill                                               Spring 1996\\
HsC 79.17: Project WILD Outdoor Education Course \hfill        Duke University

\section{FUNDING}
{\sl Graduate Student Fellowship} \hfill       2005-2007\\
Kearney Soil Science Foundation, Davis, CA \hfill        \$33,580

{\sl Mildred E. Mathias Graduate Research Grant} \hfill 2006\\
UC Natural Reserve System \hfill        \$1,000

{\sl Departmental Fellowship} \hfill 2004-2005\\
Earth System Science, UC Irvine \hfill        \$33,000

{\sl Benenson Award in the Arts} \hfill  1998\\
Duke University \hfill        \$1,500

 
\section{PUBLICATIONS} 
{\sl Peer-reviewed}

Allison, S.D, D.S. LeBauer, M.R. Ofrecio, R. Reyes, A.M. Ta, and T.M. Tran. 2008. Low levels of nitrogen addition stimulate decomposition by boreal forest fungi. Soil Bio. Biochem, 41
;293-302 

LeBauer, D.S. and K.K. Treseder. 2008. Nitrogen Limitation of Net Primary Productivity in Terrestrial Ecosystems is Globally Distributed. Ecology 89 (2): 371-379

Okano, Y., K.R. Hirstova, C.M. Leutenegger, L.E. Jackson, R.F. Denison, B. Gebreyesus, D.S. LeBauer, and K.M. Scow. (2004) Effects of ammonium on the population size of ammonia-oxidizing bacteria in soil- Application of real-time PCR. Applied and Environmental Microbiology (70) 1008-1016

\newpage
\opening

\section{PUBLICATIONS (cont.)}
{\sl Thesis chapters} \textit{in prep for submission}

 LeBauer, D.S., E.W. Seabloom, and W.S Harpole. Ecosystem responses to a gradient of nitrogen fertilization in Southern California grassland.

 LeBauer, D.S. Fungal Diversity and Decomposition in Microcosms.

{\sl Book Chapter}

 Isikhuemhen, O.S. and D.S. LeBauer. 2004. Growing \textit{Pleurotus tuberregium}. p270-281 in Mushroom Grower's Handbook 1: Oyster Mushroom Cultivation. MushWorld. Seoul, Korea

{\sl News Articles}

 LeBauer, D.S. 2008. Nitrogen Pollution Boosts Plant Growth. Scitizen.com, France

 LeBauer, D.S. 2004. Consider Shiitake Cultivation! Stewardship News, Carolina Farm Stewardship Association. Pittsboro, NC. 24(2):8-11

\section{MYCOLOGICAL CONSULTING}

{\sl Cultivation,} Oyster mushroom (\textit{Pleurotus ostreatus}) on tomato (\textit{Lycopersicon esculentum}) pomace, for Dr. Annie King and Dr. Jamal Assi, UC Davis 2003

{\sl Cultivation and Education,} Crabtree Valley Farms, Chatanooga, TN. 2000; Wildflower Organics, Dawsonville, Ga., 1999-2000; Sustenance Farm, Bear Creek, NC., 1999

{\sl Mycological Landscaping,} Sarah P. Duke Gardens, Durham, NC. Summer 1998, 1999

\section{SERVICE} {\sl Graduate Student Representative,} Department of Earth System Science, University of California at Irvine, 2006-2007

{\sl Judge,} California State Science Fair, 2006, 2007,2009

{\sl  Ad hoc reviewer,} Geophysical Research Letters, Ecology, Plant and Soil

{\sl Graduate Student Seminar Organizer,} Earth System Science, University of California at Irvine, 2005-2006

{\sl Mentor,} FARMS Leadership, Inc. 2003. 8th grade science project to investigate soil quality and mycorrhizal colonization levels in gardens. Sacramento, CA.

{\sl  Teacher,} K-12, 1998-99. Scientists 
in the Classroom, Sigma Xi, Durham, NC.

\section{PROFESSIONAL AFFILIATIONS} Ecological Society of America, Mycological Society of America, American Geophysical Union, Sigma Xi

\end{resume}
\end{document}







